%!TEX root = ts.tex

\newcommand\foralli[1][]{for all $i$ in the range of \tcode{[0, #1size())}}
\newcommand\forallmaskedi{for all selected indices $i$}
\newcommand\flagsRequires[2]{
\item If the template parameter \tcode{Flags} is \tcode{vector_aligned_tag}, \tcode{mem} shall point to storage aligned by \tcode{memory_alignment_v<#1>}.
\item If the template parameter \tcode{Flags} is \tcode{overaligned_tag<N>}, \tcode{mem} shall point to storage aligned by \tcode{N}.
\item If the template parameter \tcode{Flags} is \tcode{element_aligned_tag}, \tcode{mem} shall point to storage aligned by \tcode{alignof(#2)}.
}

\rSec0[parallel.simd]{Data-Parallel Types}
\rSec1[parallel.simd.general]{General}

\pnum
The data-parallel library consists of data-parallel types and operations on these types. A data-parallel type consists of elements of an underlying arithmetic type, called the \defn{element type}. The number of elements is a constant for each data-parallel type and called the \defn{width} of that type.

\pnum
Throughout this Clause, the term \defn{data-parallel type} refers to all \defn{supported} (\ref{parallel.simd.overview}) specializations of the \tcode{simd} and \tcode{simd_mask} class templates. A \defn{data-parallel object} is an object of \term{data-parallel type}.

\pnum
An \defn{element-wise operation} applies a specified operation to the elements of one or more data-parallel objects. Each such application is unsequenced with respect to the others. A \defn{unary element-wise operation} is an element-wise operation that applies a unary operation to each element of a data-parallel object. A \defn{binary element-wise operation} is an element-wise operation that applies a binary operation to corresponding elements of two data-parallel objects.

\pnum
Throughout this Clause, the set of \defn{vectorizable types} for a data-parallel type comprises all cv-unqualified arithmetic types other than \tcode{bool}.

\pnum
\begin{note}
The intent is to support acceleration through data-parallel execution resources, such as SIMD registers and instructions or execution units driven by a common instruction decoder. If such execution resources are unavailable, the interfaces support a transparent fallback to sequential execution.
\end{note}

\rSec1[parallel.simd.synopsis]{Header \tcode{<experimental/simd>} synopsis}

%\indexhdr{experimental/simd}
\begin{codeblock}
namespace std::experimental {
inline namespace parallelism_v2 {
  namespace simd_abi {
    using scalar = @\seebelow@;
    template<int N> using fixed_size = @\seebelow@;
    template<class T> inline constexpr int max_fixed_size = @\impdefx{maximum width supported for \tcode{fixed_size_simd} and \tcode{fixed_size_simd_mask}}@;
    template<class T> using compatible = @\impdefx{compatible and default ABI tag}@;
    template<class T> using native = @\impdefx{native ABI tag}@;

    template<class T, size_t N, class... Abis> struct deduce { using type = @\seebelow@; };
    template<class T, size_t N, class... Abis> using deduce_t =
      typename deduce<T, N, Abis...>::type;
  }

  struct element_aligned_tag {};
  struct vector_aligned_tag {};
  template<size_t> struct overaligned_tag {};
  inline constexpr element_aligned_tag element_aligned{};
  inline constexpr vector_aligned_tag vector_aligned{};
  template<size_t N> inline constexpr overaligned_tag<N> overaligned{};

  // \ref{parallel.simd.traits}, \tcode{simd} type traits
  template<class T> struct is_abi_tag;
  template<class T> inline constexpr bool is_abi_tag_v = is_abi_tag<T>::value;

  template<class T> struct is_simd;
  template<class T> inline constexpr bool is_simd_v = is_simd<T>::value;

  template<class T> struct is_simd_mask;
  template<class T> inline constexpr bool is_simd_mask_v = is_simd_mask<T>::value;

  template<class T> struct is_simd_flag_type;
  template<class T> inline constexpr bool is_simd_flag_type_v =
    is_simd_flag_type<T>::value;

  template<class T, class Abi = simd_abi::compatible<T>> struct simd_size;
  template<class T, class Abi = simd_abi::compatible<T>>
    inline constexpr size_t simd_size_v = simd_size<T,Abi>::value;

  template<class T, class U = typename T::value_type> struct memory_alignment;
  template<class T, class U = typename T::value_type>
    inline constexpr size_t memory_alignment_v = memory_alignment<T,U>::value;

  template<class T, class V> struct rebind_simd { using type = @\seebelow@; };
  template<class T, class V> using rebind_simd_t = typename rebind_simd<T, V>::type;
  template<int N, class V> struct resize_simd { using type = @\seebelow@; };
  template<int N, class V> using resize_simd_t = typename resize_simd<N, V>::type;

  // \ref{parallel.simd.class}, Class template \tcode{simd}
  template<class T, class Abi = simd_abi::compatible<T>> class simd;
  template<class T> using native_simd = simd<T, simd_abi::native<T>>;
  template<class T, int N> using fixed_size_simd = simd<T, simd_abi::fixed_size<N>>;

  // \ref{parallel.simd.mask.class}, Class template \tcode{simd_mask}
  template<class T, class Abi = simd_abi::compatible<T>> class simd_mask;
  template<class T> using native_simd_mask = simd_mask<T, simd_abi::native<T>>;
  template<class T, int N> using fixed_size_simd_mask =
    simd_mask<T, simd_abi::fixed_size<N>>;

  // \ref{parallel.simd.casts}, Casts
  template<class T, class U, class Abi> @\seebelow@ simd_cast(const simd<U, Abi>&) noexcept;
  template<class T, class U, class Abi> @\seebelow@ static_simd_cast(const simd<U, Abi>&) noexcept;

  template<class T, class Abi>
    fixed_size_simd<T, simd_size_v<T, Abi>>
      to_fixed_size(const simd<T, Abi>&) noexcept;
  template<class T, class Abi>
    fixed_size_simd_mask<T, simd_size_v<T, Abi>>
      to_fixed_size(const simd_mask<T, Abi>&) noexcept;
  template<class T, int N>
    native_simd<T> to_native(const fixed_size_simd<T, N>&) noexcept;
  template<class T, int N>
    native_simd_mask<T> to_native(const fixed_size_simd_mask<T, N>&) noexcept;
  template<class T, int N>
    simd<T> to_compatible(const fixed_size_simd<T, N>&) noexcept;
  template<class T, int N>
    simd_mask<T> to_compatible(const fixed_size_simd_mask<T, N>&) noexcept;

  template<size_t... Sizes, class T, class Abi>
    tuple<simd<T, simd_abi::deduce_t<T, Sizes>>...>
      split(const simd<T, Abi>&) noexcept;
  template<size_t... Sizes, class T, class Abi>
    tuple<simd_mask<T, simd_mask_abi::deduce_t<T, Sizes>>...>
      split(const simd_mask<T, Abi>&) noexcept;
  template<class V, class Abi>
    array<V, simd_size_v<typename V::value_type, Abi> / V::size()>
      split(const simd<typename V::value_type, Abi>&) noexcept;
  template<class V, class Abi>
    array<V, simd_size_v<typename V::simd_type::value_type, Abi> / V::size()>
      split(const simd_mask<typename V::simd_type::value_type, Abi>&) noexcept;

  template<size_t N, class T, class A>
    array<resize_simd<simd_size_v<T, A> / N, simd<T, A>>, N>
      split_by(const simd<T, A>& x) noexcept;
  template<size_t N, class T, class A>
    array<resize_simd<simd_size_v<T, A> / N, simd_mask<T, A>>, N>
      split_by(const simd_mask<T, A>& x) noexcept;

  template<class T, class... Abis>
    simd<T, simd_abi::deduce_t<T, (simd_size_v<T, Abis> + ...)>>
      concat(const simd<T, Abis>&...) noexcept;
  template<class T, class... Abis>
    simd_mask<T, simd_abi::deduce_t<T, (simd_size_v<T, Abis> + ...)>>
      concat(const simd_mask<T, Abis>&...) noexcept;

  template<class T, class Abi, size_t N>
    resize_simd<simd_size_v<T, Abi> * N, simd<T, Abi>>
      concat(const array<simd<T, Abi>, N>& arr) noexcept;
  template<class T, class Abi, size_t N>
    resize_simd<simd_size_v<T, Abi> * N, simd_mask<T, Abi>>
      concat(const array<simd_mask<T, Abi>, N>& arr) noexcept;

  // \ref{parallel.simd.mask.reductions}, Reductions
  template<class T, class Abi> bool all_of(const simd_mask<T, Abi>&) noexcept;
  template<class T, class Abi> bool any_of(const simd_mask<T, Abi>&) noexcept;
  template<class T, class Abi> bool none_of(const simd_mask<T, Abi>&) noexcept;
  template<class T, class Abi> bool some_of(const simd_mask<T, Abi>&) noexcept;
  template<class T, class Abi> int popcount(const simd_mask<T, Abi>&) noexcept;
  template<class T, class Abi> int find_first_set(const simd_mask<T, Abi>&);
  template<class T, class Abi> int find_last_set(const simd_mask<T, Abi>&);

  bool all_of(@\UNSP{T}@) noexcept;
  bool any_of(@\UNSP{T}@) noexcept;
  bool none_of(@\UNSP{T}@) noexcept;
  bool some_of(@\UNSP{T}@) noexcept;
  int popcount(@\UNSP{T}@) noexcept;
  int find_first_set(@\UNSP{T}@);
  int find_last_set(@\UNSP{T}@);

  // \ref{parallel.simd.whereexpr}, Where expression class templates
  template<class M, class T> class const_where_expression;
  template<class M, class T> class where_expression;

  // \ref{parallel.simd.mask.where}, Where functions
  template<class T, class Abi>
    where_expression<simd_mask<T, Abi>, simd<T, Abi>>
      where(const typename simd<T, Abi>::mask_type&, simd<T, Abi>&) noexcept;

  template<class T, class Abi>
    const_where_expression<simd_mask<T, Abi>, simd<T, Abi>>
      where(const typename simd<T, Abi>::mask_type&, const simd<T, Abi>&) noexcept;

  template<class T, class Abi>
    where_expression<simd_mask<T, Abi>, simd_mask<T, Abi>>
      where(const type_identity_t<simd_mask<T, Abit>>&, simd_mask<T, Abi>&) noexcept;

  template<class T, class Abi>
    const_where_expression<simd_mask<T, Abi>, simd_mask<T, Abi>>
      where(const type_identity_t<simd_mask<T, Abit>>&, const simd_mask<T, Abi>&) noexcept;

  template<class T>
    where_expression<bool, T>
      where(@\seebelow@ k, T& d) noexcept;

  template<class T>
    const_where_expression<bool, T>
      where(@\seebelow@ k, const T& d) noexcept;

  // \ref{parallel.simd.reductions}, Reductions
  template<class T, class Abi, class BinaryOperation = plus<>>
    T reduce(const simd<T, Abi>&,
             BinaryOperation = {});

  template<class M, class V, class BinaryOperation>
    typename V::value_type reduce(const const_where_expression<M, V>& x,
                                  typename V::value_type identity_element,
                                  BinaryOperation binary_op);
  template<class M, class V>
    typename V::value_type reduce(const const_where_expression<M, V>& x,
                                  plus<> binary_op = {}) noexcept;
  template<class M, class V>
    typename V::value_type reduce(const const_where_expression<M, V>& x,
                                  multiplies<> binary_op) noexcept;
  template<class M, class V>
    typename V::value_type reduce(const const_where_expression<M, V>& x,
                                  bit_and<> binary_op) noexcept;
  template<class M, class V>
    typename V::value_type reduce(const const_where_expression<M, V>& x,
                                  bit_or<> binary_op) noexcept;
  template<class M, class V>
    typename V::value_type reduce(const const_where_expression<M, V>& x,
                                  bit_xor<> binary_op) noexcept;

  template<class T, class Abi>
    T hmin(const simd<T, abi>&) noexcept;
  template<class M, class V>
    typename V::value_type hmin(const const_where_expression<M, V>&) noexcept;
  template<class T, class Abi>
    T hmax(const simd<T, abi>&) noexcept;
  template<class M, class V>
    typename V::value_type hmax(const const_where_expression<M, V>&) noexcept;

  // \ref{parallel.simd.alg}, Algorithms
  template<class T, class Abi>
    simd<T, Abi>
      min(const simd<T, Abi>& a, const simd<T, Abi>& b) noexcept;
  template<class T, class Abi>
    simd<T, Abi>
      max(const simd<T, Abi>& a, const simd<T, Abi>& b) noexcept;
  template<class T, class Abi>
    pair<simd<T, Abi>, simd<T, Abi>>
      minmax(const simd<T, Abi>& a, const simd<T, Abi>& b) noexcept;
  template<class T, class Abi>
    simd<T, Abi>
      clamp(const simd<T, Abi>& v,
            const simd<T, Abi>& lo,
            const simd<T, Abi>& hi);
}
}
\end{codeblock}

\pnum
The header \tcode{<experimental/simd>} defines class templates, tag types, trait types, and function templates for element-wise operations on data-parallel objects.

\rSec1[parallel.simd.abi]{\tcode{simd} ABI tags}

\begin{codeblock}
namespace simd_abi {
  using scalar = @\seebelow@;
  template<int N> using fixed_size = @\seebelow@;
  template<class T> inline constexpr int max_fixed_size = @\impdef@;
  template<class T> using compatible = @\impdef@;
  template<class T> using native = @\impdef@;
}
\end{codeblock}

\pnum
An \defn{ABI tag} is a type in the \tcode{std::experimental::parallelism_v2::simd_abi} namespace that indicates a choice of size and binary representation for objects of data-parallel type. \begin{note}The intent is for the size and binary representation to depend on the target architecture.\end{note} The ABI tag, together with a given element type implies a number of elements. ABI tag types are used as the second template argument to \tcode{simd} and \tcode{simd_mask}.

\pnum
\begin{note}The ABI tag is orthogonal to selecting the machine instruction set. The selected machine instruction set limits the usable ABI tag types, though (see \ref{parallel.simd.overview}). The ABI tags enable users to safely pass objects of data-parallel type between translation unit boundaries (e.g. function calls or I/O).\end{note}

\pnum
\tcode{scalar} is an alias for an unspecified ABI tag that is different from \tcode{fixed_size<1>}.
Use of the \tcode{scalar} tag type requires data-parallel types to store a single element (i.e., \tcode{simd<T, simd_abi::scalar>::size()} returns 1).

\pnum
The value of \tcode{max_fixed_size<T>} is at least 32.

\pnum
\tcode{fixed_size<N>} is an alias for an unspecified ABI tag.
\tcode{fixed_size} does not introduce a non-deduced context.
Use of the \tcode{simd_abi::fixed_size<N>} tag type requires data-parallel types to store \tcode{N} elements (i.e. \tcode{simd<T, simd_abi::fixed_size<N>>::size()} is \tcode{N}). \tcode{simd<T, fixed_size<N>>} and \tcode{simd_mask<T, fixed_size<N>>} with \tcode{N > 0} and \tcode{N <= max_fixed_size<T>} shall be supported. Additionally, for every supported \tcode{simd<T, Abi>} (see \ref{parallel.simd.overview}), where \tcode{Abi} is an ABI tag that is not a specialization of \tcode{simd_abi::fixed_size}, \tcode{N == simd<T, Abi>::size()} shall be supported.

\pnum
\begin{note}It is unspecified whether \tcode{simd<T, fixed_size<N>>} with \tcode{N > max_fixed_size<T>} is supported. The value of \tcode{max_fixed_size<T>} can depend on compiler flags and can change between different compiler versions.\end{note}

\pnum
\begin{note}An implementation can forego ABI compatibility between differently compiled translation units for \tcode{simd} and \tcode{simd_mask} specializations using the same \tcode{simd_abi::fixed_size<N>} tag. Otherwise, the efficiency of \tcode{simd<T, Abi>} is likely to be better than for \tcode{simd<T, fixed_size<simd_size_v<T, Abi>>>} (with \tcode{Abi} not a specialization of \tcode{simd_abi::fixed_size}).\end{note}

\pnum
An implementation may define additional \defn{extended ABI tag} types in the \tcode{std::experimental::parallelism_v2::simd_abi} namespace, to support other forms of data-parallel computation.

\pnum
\tcode{compatible<T>} is an implementation-defined alias for an ABI tag. \begin{note}The intent is to use the ABI tag producing the most efficient data-parallel execution for the element type \tcode{T} that ensures ABI compatibility between translation units on the target architecture.\end{note}
\begin{example}
  Consider a target architecture supporting the extended ABI tags \tcode{__simd128} and \tcode{__simd256}, where the \tcode{__simd256} type requires an optional ISA extension on said architecture. Also, the target architecture does not support \tcode{long double} with either ABI tag. The implementation therefore defines \tcode{compatible<T>} as an alias for:
  \begin{itemize}
    \item \tcode{scalar} if \tcode{T} is the same type as \tcode{long double}, and
    \item \tcode{__simd128} otherwise.
  \end{itemize}
\end{example}

\pnum
\tcode{native<T>} is an implementation-defined alias for an ABI tag. \begin{note}The intent is to use the ABI tag producing the most efficient data-parallel execution for the element type \tcode{T} that is supported on the currently targeted system. For target architectures without ISA extensions, the \tcode{native<T>} and \tcode{compatible<T>} aliases will likely be the same. For target architectures with ISA extensions, compiler flags may influence the \tcode{native<T>} alias while \tcode{compatible<T>} will be the same independent of such flags.\end{note}
\begin{example}
  Consider a target architecture supporting the extended ABI tags \tcode{__simd128} and \tcode{__simd256}, where hardware support for \tcode{__simd256} only exists for floating-point types. The implementation therefore defines \tcode{native<T>} as an alias for
  \begin{itemize}
    \item \tcode{__simd256} if \tcode{T} is a floating-point type, and
    \item \tcode{__simd128} otherwise.
  \end{itemize}
\end{example}

\begin{itemdecl}
template<T, size_t N, class... Abis> struct deduce { using type = @\seebelow@; };
\end{itemdecl}

\begin{itemdescr}
\pnum
The member \tcode{type} shall be present if and only if
\begin{itemize}
  \item \tcode{T} is a vectorizable type, and
  \item \tcode{simd_abi::fixed_size<N>} is supported (see \ref{parallel.simd.abi}), and
  \item every type in the \tcode{Abis} pack is an ABI tag.
\end{itemize}

\pnum
Where present, the member typedef \tcode{type} shall name an ABI tag type that satisfies
\begin{itemize}
  \item \tcode{simd_size<T, type> == N}, and
  \item \tcode{simd<T, type>} is default constructible (see \ref{parallel.simd.overview}).
\end{itemize}
If \tcode{N} is \tcode{1}, the member typedef \tcode{type} is \tcode{simd_abi::scalar}. Otherwise, if there are multiple ABI tag types that satisfy the constraints, the member typedef \tcode{type} is implementation-defined. \begin{note}It is expected that extended ABI tags can produce better optimizations and thus are preferred over \tcode{simd_abi::fixed_size<N>}.
Implementations can base the choice on \tcode{Abis}, but can also ignore the \tcode{Abis} arguments.
\end{note}

\pnum
The behavior of a program that adds specializations for \tcode{deduce} is undefined.
\end{itemdescr}

\rSec1[parallel.simd.traits]{\tcode{simd} type traits}

\begin{itemdecl}
template<class T> struct is_abi_tag { @\seebelow@ };
\end{itemdecl}

\begin{itemdescr}
\pnum
The type \tcode{is_abi_tag<T>} is a \tcode{UnaryTypeTrait} with a base characteristic of \tcode{true_type} if \tcode{T} is a standard or extended ABI tag, and \tcode{false_type} otherwise.

\pnum
The behavior of a program that adds specializations for \tcode{is_abi_tag} is undefined.
\end{itemdescr}

\begin{itemdecl}
template<class T> struct is_simd { @\seebelow@ };
\end{itemdecl}

\begin{itemdescr}
\pnum
The type \tcode{is_simd<T>} is a \tcode{UnaryTypeTrait} with a base characteristic of \tcode{true_type} if \tcode{T} is a specialization of the \tcode{simd} class template, and \tcode{false_type} otherwise.

\pnum
The behavior of a program that adds specializations for \tcode{is_simd} is undefined.
\end{itemdescr}

\begin{itemdecl}
template<class T> struct is_simd_mask { @\seebelow@ };
\end{itemdecl}

\begin{itemdescr}
\pnum
The type \tcode{is_simd_mask<T>} is a \tcode{UnaryTypeTrait} with a base characteristic of \tcode{true_type} if \tcode{T} is a specialization of the \tcode{simd_mask} class template, and \tcode{false_type} otherwise.

\pnum
The behavior of a program that adds specializations for \tcode{is_simd_mask} is undefined.
\end{itemdescr}

\begin{itemdecl}
template<class T> struct is_simd_flag_type { @\seebelow@ };
\end{itemdecl}

\begin{itemdescr}
\pnum
The type \tcode{is_simd_flag_type<class T>} is a \tcode{UnaryTypeTrait} with a base characteristic of \tcode{true_type} if \tcode{T} is one of
\begin{itemize}
  \item \tcode{element_aligned_tag}, or
  \item \tcode{vector_aligned_tag}, or
  \item \tcode{overaligned_tag<N>} with \tcode{N > 0} and \tcode{N} an integral power of two,
\end{itemize}
and \tcode{false_type} otherwise.

\pnum
The behavior of a program that adds specializations for \tcode{is_simd_flag_type} is undefined.
\end{itemdescr}

\begin{itemdecl}
template<class T, class Abi = simd_abi::compatible<T>> struct simd_size { @\seebelow@ };
\end{itemdecl}

\begin{itemdescr}
\pnum
\tcode{simd_size<T, Abi>} shall have a member \tcode{value} if and only if
\begin{itemize}
  \item \tcode{T} is a vectorizable type, and
  \item \tcode{is_abi_tag_v<Abi>} is \tcode{true}.
\end{itemize}
\begin{note}
  The rules are different from those in (\ref{parallel.simd.overview}).
\end{note}

\pnum
If \tcode{value} is present, the type \tcode{simd_size<T, Abi>} is a \tcode{BinaryTypeTrait} with a base characteristic of \tcode{integral_constant<size_t, N>} with \tcode{N} equal to the number of elements in a \tcode{simd<T, Abi>} object. \begin{note}If \tcode{simd<T, Abi>} is not supported for the currently targeted system, \tcode{simd_size<T, Abi>::value} produces the value \tcode{simd<T, Abi>::size()} would return if it were supported.\end{note}

\pnum
The behavior of a program that adds specializations for \tcode{simd_size} is undefined.
\end{itemdescr}

\begin{itemdecl}
template<class T, class U = typename T::value_type> struct memory_alignment { @\seebelow@ };
\end{itemdecl}

\begin{itemdescr}
\pnum
\tcode{memory_alignment<T, U>} shall have a member \tcode{value} if and only if
\begin{itemize}
  \item \tcode{is_simd_mask_v<T>} is \tcode{true} and \tcode{U} is \tcode{bool}, or
  \item \tcode{is_simd_v<T>} is \tcode{true} and \tcode{U} is a vectorizable type.
\end{itemize}

\pnum
If \tcode{value} is present, the type \tcode{memory_alignment<T, U>} is a \tcode{BinaryTypeTrait} with a base characteristic of \tcode{integral_constant<size_t, N>} for some implementation-defined \tcode{N} (see \ref{parallel.simd.copy} and \ref{parallel.simd.mask.copy}). \begin{note}\tcode{value} identifies the alignment restrictions on pointers used for (converting) loads and stores for the give type \tcode{T} on arrays of type \tcode{U}.\end{note}

\pnum
The behavior of a program that adds specializations for \tcode{memory_alignment} is undefined.
\end{itemdescr}

\begin{itemdecl}
template<class T, class V> struct rebind_simd { using type = @\seebelow@; };
\end{itemdecl}

\begin{itemdescr}
  \pnum
  The member \tcode{type} is present if and only if
  \begin{itemize}
    \item \tcode{V} is either \tcode{simd<U, Abi0>} or \tcode{simd_mask<U, Abi0>}, where \tcode{U} and \tcode{Abi0} are deduced from \tcode{V}, and
    \item \tcode{T} is a vectorizable type, and
    \item \tcode{simd_abi::deduce<T, simd_size_v<U, Abi0>, Abi0>} has a member type \tcode{type}.
  \end{itemize}

  \pnum
  Let \tcode{Abi1} denote the type \tcode{deduce_t<T, simd_size_v<U, Abi0>, Abi0>}.
  Where present, the member typedef \tcode{type} names
  \tcode{simd<T, Abi1>} if \tcode V is \tcode{simd<U, Abi0>} or
  \tcode{simd_mask<T, Abi1>} if \tcode V is \tcode{simd_mask<U, Abi0>}.
\end{itemdescr}

\begin{itemdecl}
template<int N, class V> struct resize_simd { using type = @\seebelow@; };
\end{itemdecl}

\begin{itemdescr}
  \pnum
  The member \tcode{type} is present if and only if
  \begin{itemize}
    \item \tcode{V} is either \tcode{simd<T, Abi0>} or \tcode{simd_mask<T, Abi0>}, where \tcode T and \tcode{Abi0} are deduced from \tcode V, and
    \item \tcode{simd_abi::deduce<T, N, Abi0>} has a member type \tcode{type}.
  \end{itemize}

  \pnum
  Let \tcode{Abi1} denote the type \tcode{deduce_t<T, N, Abi0>}.
  Where present, the member typedef \tcode{type} names \tcode{simd<T, Abi1>} if \tcode V is \tcode{simd<T, Abi0>} or \tcode{simd_mask<T, Abi1>} if \tcode V is \tcode{simd_mask<T, Abi0>}.
\end{itemdescr}

\rSec1[parallel.simd.whereexpr]{Where expression class templates}

\begin{codeblock}
template<class M, class T> class const_where_expression {
  const M mask;    // \expos
  T& data;         // \expos

public:
  const_where_expression(const const_where_expression&) = delete;
  const_where_expression& operator=(const const_where_expression&) = delete;

  T operator-() const && noexcept;
  T operator+() const && noexcept;
  T operator~() const && noexcept;

  template<class U, class Flags> void copy_to(U* mem, Flags f) const &&;
};

template<class M, class T>
class where_expression : public const_where_expression<M, T> {
public:
  template<class U> void operator=(U&& x) && noexcept;
  template<class U> void operator+=(U&& x) && noexcept;
  template<class U> void operator-=(U&& x) && noexcept;
  template<class U> void operator*=(U&& x) && noexcept;
  template<class U> void operator/=(U&& x) && noexcept;
  template<class U> void operator%=(U&& x) && noexcept;
  template<class U> void operator&=(U&& x) && noexcept;
  template<class U> void operator|=(U&& x) && noexcept;
  template<class U> void operator^=(U&& x) && noexcept;
  template<class U> void operator<<=(U&& x) && noexcept;
  template<class U> void operator>>=(U&& x) && noexcept;

  void operator++() && noexcept;
  void operator++(int) && noexcept;
  void operator--() && noexcept;
  void operator--(int) && noexcept;

  template<class U, class Flags> void copy_from(const U* mem, Flags) &&;
};
\end{codeblock}

\pnum
The class templates \tcode{const_where_expression} and \tcode{where_expression} abstract the notion of selecting elements of a given object of arithmetic or data-parallel type.

\pnum
The first templates argument \tcode{M} shall be cv-unqualified \tcode{bool} or a cv-unqualified \tcode{simd_mask} specialization.

\pnum
If \tcode{M} is \tcode{bool}, \tcode{T} shall be a cv-unqualified arithmetic type. Otherwise, \tcode{T} shall either be \tcode{M} or \tcode{typename M::simd_type}.

\pnum
In this subclause, if \tcode{M} is \tcode{bool}, \tcode{data[0]} is used interchangably for \tcode{data}, \tcode{mask[0]} is used interchangably for \tcode{mask}, and \tcode{M::size()} is used interchangably for \tcode{1}.

\pnum
The \defn{selected indices} signify the integers $i \in \{j \in \mathbb{N} | j < \tcode{M::size()} \wedge \tcode{mask[}j\tcode{]}\}$. The \defn{selected elements} signify the elements \tcode{data[$i$]} for all selected indices $i$.

\pnum
In this subclause, the type \tcode{value_type} is an alias for \tcode{T} if \tcode{M} is \tcode{bool}, or an alias for \tcode{typename T::value_type} if \tcode{is_simd_mask_v<M>} is \tcode{true}.

\pnum
\begin{note}
  The \tcode{where} functions \ref{parallel.simd.mask.where} initialize \tcode{mask} with the first argument to \tcode{where} and \tcode{data} with the second argument to \tcode{where}.
\end{note}

\begin{itemdecl}
T operator-() const && noexcept;
T operator+() const && noexcept;
T operator~() const && noexcept;
\end{itemdecl}

\begin{itemdescr}
  \pnum\returns
  A copy of \tcode{data} with the indicated unary operator applied to all selected elements.
\end{itemdescr}

\begin{itemdecl}
template<class U, class Flags> void copy_to(U* mem, Flags) const &&;
\end{itemdecl}

\begin{itemdescr}
  \pnum\requires
  \begin{itemize}
    \item If \tcode{M} is not \tcode{bool}, the largest selected index is less than the number of values pointed to by \tcode{mem}.
    \flagsRequires{T, U}{U}
  \end{itemize}

  \pnum\effects
  Copies the selected elements as if \tcode{mem[$i$] = static_cast<U>(data[$i$])} for all selected indices $i$.

  \pnum\throws Nothing.

  \pnum\remarks
  This function shall not participate in overload resolution unless
  \begin{itemize}
    \item \tcode{is_simd_flag_type_v<Flags>} is \tcode{true}, and
    \item either
      \begin{itemize}
        \item \tcode{U} is \tcode{bool} and \tcode{value_type} is \tcode{bool}, or
        \item \tcode{U} is a vectorizable type and \tcode{value_type} is not \tcode{bool}.
      \end{itemize}
  \end{itemize}
\end{itemdescr}

\begin{itemdecl}
template<class U> void operator=(U&& x) && noexcept;
\end{itemdecl}

\begin{itemdescr}
  \pnum\effects
  Replaces \tcode{data[$i$]} with \tcode{static_cast<T>(std\colcol{}forward<U>(x))[$i$]} for all selected indices $i$.

  \pnum\remarks
  This operator shall not participate in overload resolution unless \tcode{U} is convertible to \tcode{T}.
\end{itemdescr}

\begin{itemdecl}
template<class U> void operator+=(U&& x) && noexcept;
template<class U> void operator-=(U&& x) && noexcept;
template<class U> void operator*=(U&& x) && noexcept;
template<class U> void operator/=(U&& x) && noexcept;
template<class U> void operator%=(U&& x) && noexcept;
template<class U> void operator&=(U&& x) && noexcept;
template<class U> void operator|=(U&& x) && noexcept;
template<class U> void operator^=(U&& x) && noexcept;
template<class U> void operator<<=(U&& x) && noexcept;
template<class U> void operator>>=(U&& x) && noexcept;
\end{itemdecl}

\begin{itemdescr}
  \pnum\effects
  Replaces \tcode{data[$i$]} with \tcode{static_cast<T>(data @ std\colcol{}forward<U>(x))[$i$]} (where \tcode{@} denotes the indicated operator) for all selected indices $i$.

  \pnum\remarks
  Each of these operators shall not participate in overload resolution unless the return type of \tcode{data @ std\colcol{}forward<U>(x)} is convertible to \tcode{T}. It is unspecified whether the binary operator, implied by the compound assignment operator, is executed on all elements or only on the selected elements.
\end{itemdescr}

\begin{itemdecl}
void operator++() && noexcept;
void operator++(int) && noexcept;
void operator--() && noexcept;
void operator--(int) && noexcept;
\end{itemdecl}

\begin{itemdescr}
  \pnum\effects
  Applies the indicated operator to the selected elements.

  \pnum\remarks
  Each of these operators shall not participate in overload resolution unless the indicated operator can be applied to objects of type \tcode{T}.
\end{itemdescr}

\begin{itemdecl}
template<class U, class Flags> void copy_from(const U* mem, Flags) &&;
\end{itemdecl}

\begin{itemdescr}
  \pnum\requires
  \begin{itemize}
    \item If \tcode{is_simd_flag_type_v<U>} is \tcode{true}, for all selected indices $i$, $i$ shall be less than the number of values pointed to by \tcode{mem}.
    \flagsRequires{T, U}{U}
  \end{itemize}

  \pnum\effects
  Replaces the selected elements as if \tcode{data[$i$] = static_cast<value_type>(mem[$i$])} for all selected indices $i$.

  \pnum\throws Nothing.

  \pnum\remarks
  This function shall not participate in overload resolution unless
  \begin{itemize}
    \item \tcode{is_simd_flag_type_v<Flags>} is \tcode{true}, and
    \item either
      \begin{itemize}
        \item \tcode{U} is \tcode{bool} and \tcode{value_type} is \tcode{bool}, or
        \item \tcode{U} is a vectorizable type and \tcode{value_type} is not \tcode{bool}.
      \end{itemize}
  \end{itemize}
\end{itemdescr}

\rSec1[parallel.simd.class]{Class template \tcode{simd}}

\rSec2[parallel.simd.overview]{Class template \tcode{simd} overview}

\begin{codeblock}
template<class T, class Abi> class simd {
public:
  using value_type = T;
  using reference = @\seebelow@;
  using mask_type = simd_mask<T, Abi>;
  using abi_type = Abi;

  static constexpr size_t size() noexcept;

  simd() noexcept = default;

  // \ref{parallel.simd.ctor}, \tcode{simd} constructors
  template<class U> simd(U&& value) noexcept;
  template<class U> simd(const simd<U, simd_abi::fixed_size<size()>>&) noexcept;
  template<class G> explicit simd(G&& gen) noexcept;
  template<class U, class Flags> simd(const U* mem, Flags f);

  // \ref{parallel.simd.copy}, \tcode{simd} copy functions
  template<class U, class Flags> copy_from(const U* mem, Flags f);
  template<class U, class Flags> copy_to(U* mem, Flags f);

  // \ref{parallel.simd.subscr}, \tcode{simd} subscript operators
  reference operator[](size_t);
  value_type operator[](size_t) const;

  // \ref{parallel.simd.unary}, \tcode{simd} unary operators
  simd& operator++() noexcept;
  simd operator++(int) noexcept;
  simd& operator--() noexcept;
  simd operator--(int) noexcept;
  mask_type operator!() const noexcept;
  simd operator~() const noexcept;
  simd operator+() const noexcept;
  simd operator-() const noexcept;

  // \ref{parallel.simd.binary}, \tcode{simd} binary operators
  friend simd operator+(const simd&, const simd&) noexcept;
  friend simd operator-(const simd&, const simd&) noexcept;
  friend simd operator*(const simd&, const simd&) noexcept;
  friend simd operator/(const simd&, const simd&) noexcept;
  friend simd operator%(const simd&, const simd&) noexcept;
  friend simd operator&(const simd&, const simd&) noexcept;
  friend simd operator|(const simd&, const simd&) noexcept;
  friend simd operator^(const simd&, const simd&) noexcept;
  friend simd operator<<(const simd&, const simd&) noexcept;
  friend simd operator>>(const simd&, const simd&) noexcept;
  friend simd operator<<(const simd&, int) noexcept;
  friend simd operator>>(const simd&, int) noexcept;

  // \ref{parallel.simd.cassign}, \tcode{simd} compound assignment
  friend simd& operator+=(simd&, const simd&) noexcept;
  friend simd& operator-=(simd&, const simd&) noexcept;
  friend simd& operator*=(simd&, const simd&) noexcept;
  friend simd& operator/=(simd&, const simd&) noexcept;
  friend simd& operator%=(simd&, const simd&) noexcept;
  friend simd& operator&=(simd&, const simd&) noexcept;
  friend simd& operator|=(simd&, const simd&) noexcept;
  friend simd& operator^=(simd&, const simd&) noexcept;
  friend simd& operator<<=(simd&, const simd&) noexcept;
  friend simd& operator>>=(simd&, const simd&) noexcept;
  friend simd& operator<<=(simd&, int) noexcept;
  friend simd& operator>>=(simd&, int) noexcept;

  // \ref{parallel.simd.comparison}, \tcode{simd} compare operators
  friend mask_type operator==(const simd&, const simd&) noexcept;
  friend mask_type operator!=(const simd&, const simd&) noexcept;
  friend mask_type operator>=(const simd&, const simd&) noexcept;
  friend mask_type operator<=(const simd&, const simd&) noexcept;
  friend mask_type operator>(const simd&, const simd&) noexcept;
  friend mask_type operator<(const simd&, const simd&) noexcept;
};
\end{codeblock}

\pnum
The class template \tcode{simd} is a data-parallel type. The width of a given \tcode{simd} specialization is a constant expression, determined by the template parameters.

\pnum
Every specialization of \tcode{simd} shall be a complete type. The specialization \tcode{simd<T, Abi>} is supported if \tcode{T} is a vectorizable type and
\begin{itemize}
  \item \tcode{Abi} is \tcode{simd_abi::scalar}, or
  \item \tcode{Abi} is \tcode{simd_abi::fixed_size<N>}, with \tcode{N} constrained as defined in \ref{parallel.simd.abi}.
\end{itemize}

If \tcode{Abi} is an extended ABI tag, it is implementation-defined whether \tcode{simd<T, Abi>} is supported. \begin{note}The intent is for implementations to decide on the basis of the currently targeted system.\end{note}

If \tcode{simd<T, Abi>} is not supported, the specialization shall have a deleted default constructor, deleted destructor, deleted copy constructor, and deleted copy assignment.
Otherwise, the following are true:
\begin{itemize}
  \item \tcode{is_nothrow_move_constructible_v<simd<T, Abi>>}, and
  \item \tcode{is_nothrow_move_assignable_v<simd<T, Abi>>}, and
  \item \tcode{is_nothrow_default_constructible_v<simd<T, Abi>>}.
\end{itemize}

\begin{example}
  Consider an implementation that defines the extended ABI tags \tcode{__simd_x} and \tcode{__gpu_y}. When the compiler is invoked to translate to a machine that has support for the \tcode{__simd_x} ABI tag for all arithmetic types other than \tcode{long double} and no support for the \tcode{__gpu_y} ABI tag, then:
  \begin{itemize}
    \item \tcode{simd<T, simd_abi::__gpu_y>} is not supported for any \tcode{T} and has a deleted constructor.
    \item \tcode{simd<long double, simd_abi::__simd_x>} is not supported and has a deleted constructor.
    \item \tcode{simd<double, simd_abi::__simd_x>} is supported.
    \item \tcode{simd<long double, simd_abi::scalar>} is supported.
  \end{itemize}
\end{example}

\pnum
Default intialization performs no initialization of the elements; value-initialization initializes each element with \tcode{T()}. \begin{note}Thus, default initialization leaves the elements in an indeterminate state.\end{note}

\pnum
Implementations should enable explicit conversion from and to implementation-defined types. This adds one or more of the following declarations to class \tcode{simd}:

\begin{codeblock}
explicit operator @\impdefx{type(s) \tcode{simd} is explicitly convertible to}@() const;
explicit simd(const @\impdefx{type(s) \tcode{simd} is explicit convertible from}@& init);
\end{codeblock}

\begin{example}
  Consider an implementation that supports the type \tcode{__vec4f} and the function \tcode{__vec4f _vec4f_addsub(__vec4f, __vec4f)} for the currently targeted system.
  A user may require the use of \tcode{_vec4f_addsub} for maximum performance and thus writes:
  \begin{codeblock}
    using V = simd<float, simd_abi::__simd128>;
    V addsub(V a, V b) {
      return static_cast<V>(_vec4f_addsub(static_cast<__vec4f>(a), static_cast<__vec4f>(b)));
    }
  \end{codeblock}
\end{example}

\rSec2[parallel.simd.width]{\tcode{simd} width}

\begin{itemdecl}
static constexpr size_t size() noexcept;
\end{itemdecl}

\begin{itemdescr}
  \pnum\returns
  The width of \tcode{simd<T, Abi>}.
\end{itemdescr}

\rSec2[parallel.simd.reference]{Element references}

\pnum
A \tcode{reference} is an object that refers to an element in a \tcode{simd} or \tcode{simd_mask} object. \tcode{reference::value_type} is the same type as \tcode{simd::value_type} or \tcode{simd_mask::value_type}, respectively.

\pnum
Class \tcode{reference} is for exposition only. An implementation is permitted to provide equivalent functionality without providing a class with this name.

\begin{codeblock}
class reference // \expos
{
public:
  reference() = delete;
  reference(const reference&) = delete;

  operator value_type() const noexcept;

  template<class U> reference operator=(U&& x) && noexcept;

  template<class U> reference operator+=(U&& x) && noexcept;
  template<class U> reference operator-=(U&& x) && noexcept;
  template<class U> reference operator*=(U&& x) && noexcept;
  template<class U> reference operator/=(U&& x) && noexcept;
  template<class U> reference operator%=(U&& x) && noexcept;
  template<class U> reference operator|=(U&& x) && noexcept;
  template<class U> reference operator&=(U&& x) && noexcept;
  template<class U> reference operator^=(U&& x) && noexcept;
  template<class U> reference operator<<=(U&& x) && noexcept;
  template<class U> reference operator>>=(U&& x) && noexcept;

  reference operator++() && noexcept;
  value_type operator++(int) && noexcept;
  reference operator--() && noexcept;
  value_type operator--(int) && noexcept;

  friend void swap(reference&& a, reference&& b) noexcept;
  friend void swap(value_type&& a, reference&& b) noexcept;
  friend void swap(reference&& a, value_type&& b) noexcept;
};
\end{codeblock}

\begin{itemdecl}
operator value_type() const noexcept;
\end{itemdecl}

\begin{itemdescr}
  \pnum\returns
  The value of the element referred to by \tcode{*this}.
\end{itemdescr}

\begin{itemdecl}
template<class U> reference operator=(U&& x) && noexcept;
\end{itemdecl}

\begin{itemdescr}
  \pnum\effects
  Replaces the referred to element in \tcode{simd} or \tcode{simd_mask} with \tcode{static_cast<value_type>(std\colcol{}forward<U>(x))}.

  \pnum\returns
  A copy of \tcode{*this}.

  \pnum\remarks
  This function shall not participate in overload resolution unless \tcode{declval<value_type\&>() = std\colcol{}forward>U>(x)} is well-formed.
\end{itemdescr}

\begin{itemdecl}
template<class U> reference operator+=(U&& x) && noexcept;
template<class U> reference operator-=(U&& x) && noexcept;
template<class U> reference operator*=(U&& x) && noexcept;
template<class U> reference operator/=(U&& x) && noexcept;
template<class U> reference operator%=(U&& x) && noexcept;
template<class U> reference operator|=(U&& x) && noexcept;
template<class U> reference operator&=(U&& x) && noexcept;
template<class U> reference operator^=(U&& x) && noexcept;
template<class U> reference operator<<=(U&& x) && noexcept;
template<class U> reference operator>>=(U&& x) && noexcept;
\end{itemdecl}

\begin{itemdescr}
  \pnum\effects
  Applies the indicated compound operator to the referred to element in \tcode{simd} or \tcode{simd_mask} and \tcode{std\colcol{}forward<U>(x)}.

  \pnum\returns
  A copy of \tcode{*this}.

  \pnum\remarks
  This function shall not participate in overload resolution unless \tcode{declval<value_type\&>() @= std\colcol{}forward<U>(x)} (where \tcode{@=} denotes the indicated compound assignment operator) is well-formed.
\end{itemdescr}

\begin{itemdecl}
reference operator++() && noexcept;
reference operator--() && noexcept;
\end{itemdecl}

\begin{itemdescr}
  \pnum\effects
  Applies the indicated operator to the referred to element in \tcode{simd} or \tcode{simd_mask}.

  \pnum\returns
  A copy of \tcode{*this}.

  \pnum\remarks
  This function shall not participate in overload resolution unless the indicated operator can be applied to objects of type \tcode{value_type}.
\end{itemdescr}

\begin{itemdecl}
value_type operator++(int) && noexcept;
value_type operator--(int) && noexcept;
\end{itemdecl}

\begin{itemdescr}
  \pnum\effects
  Applies the indicated operator to the referred to element in \tcode{simd} or \tcode{simd_mask}.

  \pnum\returns
  A copy of the referred to element before applying the indicated operator.

  \pnum\remarks
  This function shall not participate in overload resolution unless the indicated operator can be applied to objects of type \tcode{value_type}.
\end{itemdescr}

\begin{itemdecl}
friend void swap(reference&& a, reference&& b) noexcept;
friend void swap(value_type& a, reference&& b) noexcept;
friend void swap(reference&& a, value_type& b) noexcept;
\end{itemdecl}

\begin{itemdescr}
  \pnum\effects
  Exchanges the values \tcode{a} and \tcode{b} refer to.
\end{itemdescr}

\rSec2[parallel.simd.ctor]{\tcode{simd} constructors}

\begin{itemdecl}
template<class U> simd(U&&) noexcept;
\end{itemdecl}

\begin{itemdescr}
  \pnum\effects
  Constructs an object with each element initialized to the value of the argument after conversion to \tcode{value_type}.

  \pnum\remarks
  Let \tcode{From} denote the type \tcode{remove_cv_t<remove_reference_t<U>>}. This constructor shall not participate in overload resolution unless:
  \begin{itemize}
    \item \tcode{From} is a vectorizable type and every possibly value of \tcode{From} can be represented with type \tcode{value_type}, or
    \item \tcode{From} is not an arithmetic type and is implicitly convertible to \tcode{value_type}, or
    \item \tcode{From} is \tcode{int}, or
    \item \tcode{From} is \tcode{unsigned int} and \tcode{value_type} is an unsigned integral type.
  \end{itemize}

\end{itemdescr}

\begin{itemdecl}
template<class U> simd(const simd<U, simd_abi::fixed_size<size()>>& x) noexcept;
\end{itemdecl}

\begin{itemdescr}
  \pnum\effects
  Constructs an object where the $i^\text{th}$ element equals \tcode{static_cast<T>(x[$i$])} \foralli.

  \pnum\remarks
  This constructor shall not participate in overload resolution unless
  \begin{itemize}
    \item \tcode{abi_type} is \tcode{simd_abi::fixed_size<size()>}, and
    \item every possible value of \tcode{U} can be represented with type \tcode{value_type}, and
    \item if both \tcode{U} and \tcode{value_type} are integral, the integer conversion rank (\CppXref{conv.rank}) of \tcode{value_type} is greater than the integer conversion rank of \tcode{U}.
  \end{itemize}
\end{itemdescr}

\begin{itemdecl}
template<class G> simd(G&& gen) noexcept;
\end{itemdecl}

\begin{itemdescr}
  \pnum\effects
  Constructs an object where the $i^\text{th}$ element is initialized to \tcode{gen(integral_constant<size_t, i>())}.

  \pnum\remarks
  This constructor shall not participate in overload resolution unless \tcode{simd(gen(integral_constant<size_t, i>()))} is well-formed \foralli.

  \pnum
  The calls to \tcode{gen} are unsequenced with respect to each other. Vectorization-unsafe standard library functions may not be invoked by \tcode{gen} (\CppXref{algorithms.parallel.exec}).
\end{itemdescr}

\begin{itemdecl}
template<class U, class Flags> simd(const U* mem, Flags);
\end{itemdecl}

\begin{itemdescr}
  \pnum\requires
  \begin{itemize}
    \item \tcode{[mem, mem + size())} is a valid range.
    \flagsRequires{simd, U}{U}
  \end{itemize}

  \pnum\effects
  Constructs an object where the $i^\text{th}$ element is initialized to \tcode{static_cast<T>(mem[$i$])} \foralli.

  \pnum\remarks
  This constructor shall not participate in overload resolution unless
  \begin{itemize}
    \item \tcode{is_simd_flag_type_v<Flags>} is \tcode{true}, and
    \item \tcode{U} is a vectorizable type.
  \end{itemize}
\end{itemdescr}

\rSec2[parallel.simd.copy]{\tcode{simd} copy functions}

\begin{itemdecl}
template<class U, class Flags> void copy_from(const U* mem, Flags);
\end{itemdecl}

\begin{itemdescr}
  \pnum\requires
  \begin{itemize}
    \item \tcode{[mem, mem + size())} is a valid range.
    \flagsRequires{simd, U}{U}
  \end{itemize}

  \pnum\effects
  Replaces the elements of the \tcode{simd} object such that the $i^\text{th}$ element is assigned with \tcode{static_cast<T>(mem[$i$])} \foralli.

  \pnum\remarks
  This function shall not participate in overload resolution unless
  \begin{itemize}
    \item \tcode{is_simd_flag_type_v<Flags>} is \tcode{true}, and
    \item \tcode{U} is a vectorizable type.
  \end{itemize}
\end{itemdescr}

\begin{itemdecl}
template<class U, class Flags> void copy_to(U* mem, Flags) const;
\end{itemdecl}

\begin{itemdescr}
  \pnum\requires
  \begin{itemize}
    \item \tcode{[mem, mem + size())} is a valid range.
    \flagsRequires{simd, U}{U}
  \end{itemize}

  \pnum\effects
  Copies all \tcode{simd} elements as if \tcode{mem[$i$] = static_cast<U>(operator[]($i$))} \foralli.

  \pnum\remarks
  This function shall not participate in overload resolution unless
  \begin{itemize}
    \item \tcode{is_simd_flag_type_v<Flags>} is \tcode{true}, and
    \item \tcode{U} is a vectorizable type.
  \end{itemize}
\end{itemdescr}

\rSec2[parallel.simd.subscr]{\tcode{simd} subscript operators}

\begin{itemdecl}
reference operator[](size_t i);
\end{itemdecl}

\begin{itemdescr}
  \pnum\requires
  \tcode{i < size()}.

  \pnum\returns
  A \tcode{reference} (see \ref{parallel.simd.reference}) referring to the $i^\text{th}$ element.

  \pnum\throws Nothing.
\end{itemdescr}

\begin{itemdecl}
value_type operator[](size_t i) const;
\end{itemdecl}

\begin{itemdescr}
  \pnum\requires
  \tcode{i < size()}.

  \pnum\returns
  The value of the $i^\text{th}$ element.

  \pnum\throws Nothing.
\end{itemdescr}

\rSec2[parallel.simd.unary]{\tcode{simd} unary operators}

\pnum
Effects in this subclause are applied as unary element-wise operations.

\begin{itemdecl}
simd& operator++() noexcept;
\end{itemdecl}

\begin{itemdescr}
  \pnum\effects
  Increments every element by one.

  \pnum\returns
  \tcode{*this}.
\end{itemdescr}

\begin{itemdecl}
simd operator++(int) noexcept;
\end{itemdecl}

\begin{itemdescr}
  \pnum\effects
  Increments every element by one.

  \pnum\returns
  A copy of \tcode{*this} before incrementing.
\end{itemdescr}

\begin{itemdecl}
simd& operator--() noexcept;
\end{itemdecl}

\begin{itemdescr}
  \pnum\effects
  Decrements every element by one.

  \pnum\returns
  \tcode{*this}.
\end{itemdescr}

\begin{itemdecl}
simd operator--(int) noexcept;
\end{itemdecl}

\begin{itemdescr}
  \pnum\effects
  Decrements every element by one.

  \pnum\returns
  A copy of \tcode{*this} before decrementing.
\end{itemdescr}

\begin{itemdecl}
mask_type operator!() const noexcept;
\end{itemdecl}

\begin{itemdescr}
  \pnum\returns
  A \tcode{simd_mask} object with the $i^\text{th}$ element set to \tcode{!operator[]($i$)} \foralli.
\end{itemdescr}

\begin{itemdecl}
simd operator~() const noexcept;
\end{itemdecl}

\begin{itemdescr}
  \pnum\returns
  A \tcode{simd} object where each bit is the inverse of the corresponding bit in \tcode{*this}.

  \pnum\remarks
  This operator shall not participate in overload resolution unless \tcode{T} is an integral type.
\end{itemdescr}

\begin{itemdecl}
simd operator+() const noexcept;
\end{itemdecl}

\begin{itemdescr}
  \pnum\returns
  \tcode{*this}.
\end{itemdescr}

\begin{itemdecl}
simd operator-() const noexcept;
\end{itemdecl}

\begin{itemdescr}
  \pnum\returns
  A \tcode{simd} object where the $i^\text{th}$ element is initialized to \tcode{-operator[]($i$)} \foralli.
\end{itemdescr}

\rSec1[parallel.simd.nonmembers]{\tcode{simd} non-member operations}

\rSec2[parallel.simd.binary]{\tcode{simd} binary operators}

\begin{itemdecl}
friend simd operator+(const simd& lhs, const simd& rhs) noexcept;
friend simd operator-(const simd& lhs, const simd& rhs) noexcept;
friend simd operator*(const simd& lhs, const simd& rhs) noexcept;
friend simd operator/(const simd& lhs, const simd& rhs) noexcept;
friend simd operator%(const simd& lhs, const simd& rhs) noexcept;
friend simd operator&(const simd& lhs, const simd& rhs) noexcept;
friend simd operator|(const simd& lhs, const simd& rhs) noexcept;
friend simd operator^(const simd& lhs, const simd& rhs) noexcept;
friend simd operator<<(const simd& lhs, const simd& rhs) noexcept;
friend simd operator>>(const simd& lhs, const simd& rhs) noexcept;
\end{itemdecl}

\begin{itemdescr}
  \pnum\returns
  A \tcode{simd} object initialized with the results of applying the indicated operator to \tcode{lhs} and \tcode{rhs} as a binary element-wise operation.

  \pnum\remarks
  Each of these operators shall not participate in overload resolution unless the indicated operator can be applied to objects of type \tcode{value_type}.
\end{itemdescr}

\begin{itemdecl}
friend simd operator<<(const simd& v, int n) noexcept;
friend simd operator>>(const simd& v, int n) noexcept;
\end{itemdecl}

\begin{itemdescr}
  \pnum\returns
  A \tcode{simd} object where the $i^\text{th}$ element is initialized to the result of applying the indicated operator to \tcode{v[$i$]} and \tcode{n} \foralli.

  \pnum\remarks
  These operators shall not participate in overload resolution unless the indicated operator can be applied to objects of type \tcode{value_type}.
\end{itemdescr}

\rSec2[parallel.simd.cassign]{\tcode{simd} compound assignment}

\begin{itemdecl}
friend simd& operator+=(simd& lhs, const simd& rhs) noexcept;
friend simd& operator-=(simd& lhs, const simd& rhs) noexcept;
friend simd& operator*=(simd& lhs, const simd& rhs) noexcept;
friend simd& operator/=(simd& lhs, const simd& rhs) noexcept;
friend simd& operator%=(simd& lhs, const simd& rhs) noexcept;
friend simd& operator&=(simd& lhs, const simd& rhs) noexcept;
friend simd& operator|=(simd& lhs, const simd& rhs) noexcept;
friend simd& operator^=(simd& lhs, const simd& rhs) noexcept;
friend simd& operator<<=(simd& lhs, const simd& rhs) noexcept;
friend simd& operator>>=(simd& lhs, const simd& rhs) noexcept;
\end{itemdecl}

\begin{itemdescr}
  \pnum\effects
  These operators apply the indicated operator to \tcode{lhs} and \tcode{rhs} as an element-wise operation.

  \pnum\returns
  \tcode{lhs}.

  \pnum\remarks
  These operators shall not participate in overload resolution unless the indicated operator can be applied to objects of type \tcode{value_type}.
\end{itemdescr}

\begin{itemdecl}
friend simd& operator<<=(simd& lhs, int n) noexcept;
friend simd& operator>>=(simd& lhs, int n) noexcept;
\end{itemdecl}

\begin{itemdescr}
  \pnum\effects
  Equivalent to: \tcode{return operator@=(lhs, simd(n));}

  \pnum\remarks
  These operators shall not participate in overload resolution unless the indicated operator can be applied to objects of type \tcode{value_type}.
\end{itemdescr}

\rSec2[parallel.simd.comparison]{\tcode{simd} compare operators}

\begin{itemdecl}
friend mask_type operator==(const simd& lhs, const simd& rhs) noexcept;
friend mask_type operator!=(const simd& lhs, const simd& rhs) noexcept;
friend mask_type operator>=(const simd& lhs, const simd& rhs) noexcept;
friend mask_type operator<=(const simd& lhs, const simd& rhs) noexcept;
friend mask_type operator>(const simd& lhs, const simd& rhs) noexcept;
friend mask_type operator<(const simd& lhs, const simd& rhs) noexcept;
\end{itemdecl}

\begin{itemdescr}
  \pnum\returns
  A \tcode{simd_mask} object initialized with the results of applying the indicated operator to \tcode{lhs} and \tcode{rhs} as a binary element-wise operation.
\end{itemdescr}

\rSec2[parallel.simd.reductions]{Reductions}

\pnum
In this subclause, \tcode{BinaryOperation} shall be a binary element-wise operation.

\begin{itemdecl}
template<class T, class Abi, class BinaryOperation = plus<>>
  T reduce(const simd<T, Abi>& x, BinaryOperation binary_op = {});
\end{itemdecl}

\begin{itemdescr}
  \pnum\requires
  \tcode{binary_op} shall be callable with two arguments of type \tcode{T} returning \tcode{T}, or callable with two arguments of type \tcode{simd<T, A1>} returning \tcode{simd<T, A1>} for every \tcode{A1} that is an ABI tag type.

  \pnum\returns
  \tcode{\placeholdernc{GENERALIZED_SUM}(binary_op, x.data[$i$], ...)} \foralli (\CppXref{numerics.defns}).

  \pnum\throws
  Any exception thrown from \tcode{binary_op}.

\end{itemdescr}

\begin{itemdecl}
template<class M, class V, class BinaryOperation>
  typename V::value_type reduce(const const_where_expression<M, V>& x,
                                typename V::value_type identity_element,
                                BinaryOperation binary_op = {});
\end{itemdecl}

\begin{itemdescr}
  \pnum\requires
  \tcode{binary_op} shall be callable with two arguments of type \tcode{T} returning \tcode{T}, or callable with two arguments of type \tcode{simd<T, A1>} returning \tcode{simd<T, A1>} for every \tcode{A1} that is an ABI tag type. The results of \tcode{binary_op(identity_element, x)} and \tcode{binary_op(x, identity_element)} shall be equal to \tcode{x} for all finite values \tcode{x} representable by \tcode{V::value_type}.

  \pnum\returns
  If \tcode{none_of(x.mask)}, returns \tcode{identity_element}. Otherwise, returns \tcode{\placeholdernc{GENERALIZED_SUM}(binary_op, x.data[$i$], ...)} \forallmaskedi.

  \pnum\throws
  Any exception thrown from \tcode{binary_op}.
\end{itemdescr}

\begin{itemdecl}
template<class M, class V>
  typename V::value_type reduce(const const_where_expression<M, V>& x, plus<> binary_op) noexcept;
\end{itemdecl}

\begin{itemdescr}
  \pnum\returns
  If \tcode{none_of(x.mask)}, returns \tcode{0}. Otherwise, returns \tcode{\placeholdernc{GENERALIZED_SUM}(binary_op, x.data[$i$], ...)} \forallmaskedi.
\end{itemdescr}

\begin{itemdecl}
template<class M, class V>
  typename V::value_type reduce(const const_where_expression<M, V>& x, multiplies<> binary_op) noexcept;
\end{itemdecl}

\begin{itemdescr}
  \pnum\returns
  If \tcode{none_of(x.mask)}, returns \tcode{1}. Otherwise, returns \tcode{\placeholdernc{GENERALIZED_SUM}(binary_op, x.data[$i$], ...)} \forallmaskedi.
\end{itemdescr}

\begin{itemdecl}
template<class M, class V>
  typename V::value_type reduce(const const_where_expression<M, V>& x, bit_and<> binary_op) noexcept;
\end{itemdecl}

\begin{itemdescr}
  \pnum\requires
  \tcode{is_integral_v<V::value_type>} is \tcode{true}.

  \pnum\returns
  If \tcode{none_of(x.mask)}, returns \tcode{\~{}V::value_type()}. Otherwise, returns \tcode{\placeholdernc{GENERALIZED_SUM}(binary_op, x.data[$i$], ...)} \forallmaskedi.
\end{itemdescr}

\begin{itemdecl}
template<class M, class V>
  typename V::value_type reduce(const const_where_expression<M, V>& x, bit_or<> binary_op) noexcept;
template<class M, class V>
  typename V::value_type reduce(const const_where_expression<M, V>& x, bit_xor<> binary_op) noexcept;
\end{itemdecl}

\begin{itemdescr}
  \pnum\requires
  \tcode{is_integral_v<V::value_type>} is \tcode{true}.

  \pnum\returns
  If \tcode{none_of(x.mask)}, returns \tcode{0}. Otherwise, returns \tcode{\placeholdernc{GENERALIZED_SUM}(binary_op, x.data[$i$], ...)} \forallmaskedi.
\end{itemdescr}

\begin{itemdecl}
template<class T, class Abi> T hmin(const simd<T, Abi>& x) noexcept;
\end{itemdecl}

\begin{itemdescr}
  \pnum\returns
  The value of an element \tcode{x[$j$]} for which \tcode{x[$j$] <= x[$i$]} \foralli.
\end{itemdescr}

\begin{itemdecl}
template<class M, class V> typename V::value_type hmin(const const_where_expression<M, V>& x) noexcept;
\end{itemdecl}

\begin{itemdescr}
  \pnum\returns
  If \tcode{none_of(x.mask)}, the return value is \tcode{numeric_limits<V::value_type>::max()}. Otherwise, returns the value of an element \tcode{x.data[$j$]} for which \tcode{x.mask[$j$] == true} and \tcode{x.data[$j$] <= x.data[$i$]} \forallmaskedi.
\end{itemdescr}

\begin{itemdecl}
template<class T, class Abi> T hmax(const simd<T, Abi>& x) noexcept;
\end{itemdecl}

\begin{itemdescr}
  \pnum\returns
  The value of an element \tcode{x[$j$]} for which \tcode{x[$j$] >= x[$i$]} \foralli.
\end{itemdescr}

\begin{itemdecl}
template<class M, class V> typename V::value_type hmax(const const_where_expression<M, V>& x) noexcept;
\end{itemdecl}

\begin{itemdescr}
  \pnum\returns
  If \tcode{none_of(x.mask)}, the return value is \tcode{numeric_limits<V::value_type>::lowest()}. Otherwise, returns the value of an element \tcode{x.data[$j$]} for which \tcode{x.mask[$j$] == true} and \tcode{x.data[$j$] >= x.data[$i$]} \forallmaskedi.
\end{itemdescr}

\rSec2[parallel.simd.casts]{Casts}

\begin{itemdecl}
template<class T, class U, class Abi> @\seebelow@ simd_cast(const simd<U, Abi>& x) noexcept;
\end{itemdecl}

\begin{itemdescr}
  \pnum
  Let \tcode{To} denote \tcode{T::value_type} if \tcode{is_simd_v<T>} is \tcode{true}, or \tcode{T} otherwise.

  \pnum\returns
  A \tcode{simd} object with the $i^\text{th}$ element initialized to \tcode{static_cast<To>(x[$i$])} \foralli.

  \pnum\remarks
  The function shall not participate in overload resolution unless
  \begin{itemize}
    \item every possible value of type \tcode{U} can be represented with type \tcode{To}, and
    \item either
      \begin{itemize}
        \item \tcode{is_simd_v<T>} is \tcode{false}, or
        \item \tcode{T::size() == simd<U, Abi>::size()} is \tcode{true}.
      \end{itemize}
  \end{itemize}

  \pnum
  The return type is
  \begin{itemize}
    \item \tcode{T} if \tcode{is_simd_v<T>} is \tcode{true};
    \item otherwise, \tcode{simd<T, Abi>} if \tcode{U} is the same type as \tcode{T};
    \item otherwise, \tcode{simd<T, simd_abi::fixed_size<simd<U, Abi>::size()>>}
  \end{itemize}
\end{itemdescr}

\begin{itemdecl}
template<class T, class U, class Abi> @\seebelow@ static_simd_cast(const simd<U, Abi>& x) noexcept;
\end{itemdecl}

\begin{itemdescr}
  \pnum
  Let \tcode{To} denote \tcode{T::value_type} if \tcode{is_simd_v<T>} is \tcode{true} or \tcode{T} otherwise.

  \pnum\returns
  A \tcode{simd} object with the $i^\text{th}$ element initialized to \tcode{static_cast<To>(x[$i$])} \foralli.

  \pnum\remarks
  The function shall not participate in overload resolution unless either
  \begin{itemize}
    \item \tcode{is_simd_v<T>} is \tcode{false}, or
    \item \tcode{T::size() == simd<U, Abi>::size()} is \tcode{true}.
  \end{itemize}

  \pnum
  The return type is
  \begin{itemize}
    \item \tcode{T} if \tcode{is_simd_v<T>} is \tcode{true};
    \item otherwise, \tcode{simd<T, Abi>} if either \tcode{U} is the same type as \tcode{T} or \tcode{make_signed_t<U>} is the same type as \tcode{make_signed_t<T>};
    \item otherwise, \tcode{simd<T, simd_abi::fixed_size<simd<U, Abi>::size()>>}.
  \end{itemize}
\end{itemdescr}

\begin{itemdecl}
template<class T, class Abi>
  fixed_size_simd<T, simd_size_v<T, Abi>> to_fixed_size(const simd<T, Abi>& x) noexcept;
template<class T, class Abi>
  fixed_size_simd_mask<T, simd_size_v<T, Abi>> to_fixed_size(const simd_mask<T, Abi>& x) noexcept;
\end{itemdecl}

\begin{itemdescr}
  \pnum\returns
  A data-parallel object with the $i^\text{th}$ element initialized to \tcode{x[$i$]} \foralli.
\end{itemdescr}

\begin{itemdecl}
template<class T, int N> native_simd<T> to_native(const fixed_size_simd<T, N>& x) noexcept;
template<class T, int N> native_simd_mask<T> to_native(const fixed_size_simd_mask<T, N>& x) noexcept;
\end{itemdecl}

\begin{itemdescr}
  \pnum\returns
  A data-parallel object with the $i^\text{th}$ element initialized to \tcode{x[$i$]} \foralli.

  \pnum\remarks
  These functions shall not participate in overload resolution unless \tcode{simd_size_v<T, simd_abi::native<T>> == N} is \tcode{true}.
\end{itemdescr}

\begin{itemdecl}
template<class T, int N> simd<T> to_compatible(const fixed_size_simd<T, N>& x) noexcept;
template<class T, int N> simd_mask<T> to_compatible(const fixed_size_simd_mask<T, N>& x) noexcept;
\end{itemdecl}

\begin{itemdescr}
  \pnum\returns
  A data-parallel object with the $i^\text{th}$ element initialized to \tcode{x[$i$]} \foralli.

  \pnum\remarks
  These functions shall not participate in overload resolution unless \tcode{simd_size_v<T, simd_abi::compatible<T>> == N} is \tcode{true}.
\end{itemdescr}

\begin{itemdecl}
template<size_t... Sizes, class T, class Abi>
  tuple<simd<T, simd_abi::deduce_t<T, Sizes>>...>
    split(const simd<T, Abi>& x) noexcept;
template<size_t... Sizes, class T, class Abi>
  tuple<simd_mask<T, simd_abi::deduce_t<T, Sizes>>...>
    split(const simd_mask<T, Abi>& x) noexcept;
\end{itemdecl}

\begin{itemdescr}
  \pnum\returns
  A \tcode{tuple} of data-parallel objects with the $i^\text{th}$ \tcode{simd}/\tcode{simd_mask} element of the $j^\text{th}$ \tcode{tuple} element initialized to the value of the element \tcode{x} with index $i$ + sum of the first $j$ values in the \tcode{Sizes} pack.

  \pnum\remarks
  These functions shall not participate in overload resolution unless the sum of all values in the \tcode{Sizes} pack is equal to \tcode{simd_size_v<T, Abi>}.
\end{itemdescr}

\begin{itemdecl}
template<class V, class Abi>
  array<V, simd_size_v<typename V::value_type, Abi> / V::size()>
    split(const simd<typename V::value_type, Abi>& x) noexcept;
template<class V, class Abi>
  array<V, simd_size_v<typename V::simd_type::value_type, Abi> / V::size()>
    split(const simd_mask<typename V::simd_type::value_type, Abi>& x) noexcept;
\end{itemdecl}

\begin{itemdescr}
  \pnum\returns
  An \tcode{array} of data-parallel objects with the $i^\text{th}$ \tcode{simd}/\tcode{simd_mask} element of the $j^\text{th}$ \tcode{array} element initialized to the value of the element in \tcode{x} with index \tcode{$i$ + $j$ * V::size()}.

  \pnum\remarks
  These functions shall not participate in overload resolution unless either:
  \begin{itemize}
    \item \tcode{is_simd_v<V>} is \tcode{true} and \tcode{simd_size_v<typename V::value_type, Abi>} is an integral multiple of \tcode{V::size()}, or
    \item \tcode{is_simd_mask_v<V>} is \tcode{true} and \tcode{simd_size_v<typename V::simd_type::value_type, Abi>} is an integral multiple of \tcode{V::size()}.
  \end{itemize}
\end{itemdescr}

\begin{itemdecl}
template<size_t N, class T, class A>
  array<resize_simd<simd_size_v<T, A> / N, simd<T, A>>, N>
    split_by(const simd<T, A>& x) noexcept;
template<size_t N, class T, class A>
  array<resize_simd<simd_size_v<T, A> / N, simd_mask<T, A>>, N>
    split_by(const simd_mask<T, A>& x) noexcept;
\end{itemdecl}

\begin{itemdescr}
  \pnum\returns
  An array \tcode{arr}, where \tcode{arr[$i$][$j$]} is initialized by \tcode{x[$i$ * (simd_size_v<T, A> / N) + $j$]}.

  \pnum\remarks
  The functions shall not participate in overload resolution unless \tcode{simd_size_v<T, A>} is an integral multiple of \tcode N.
\end{itemdescr}

\begin{itemdecl}
template<class T, class... Abis>
  simd<T, simd_abi::deduce_t<T, (simd_size_v<T, Abis> + ...)>> concat(
    const simd<T, Abis>&... xs) noexcept;
template<class T, class... Abis>
  simd_mask<T, simd_abi::deduce_t<T, (simd_size_v<T, Abis> + ...)>> concat(
    const simd_mask<T, Abis>&... xs) noexcept;
\end{itemdecl}

\begin{itemdescr}
  \pnum\returns
  A data-parallel object initialized with the concatenated values in the \tcode{xs} pack of data-parallel objects: The $i^\text{th}$ \tcode{simd}/\tcode{simd_mask} element of the $j^\text{th}$ parameter in the \tcode{xs} pack is copied to the return value's element with index $i$ + the sum of the width of the first $j$ parameters in the \tcode{xs} pack.
\end{itemdescr}

\begin{itemdecl}
  template<class T, class Abi, size_t N>
    resize_simd<simd_size_v<T, Abi> * N, simd<T, Abi>>
      concat(const array<simd<T, Abi>, N>& arr) noexcept;
  template<class T, class Abi, size_t N>
    resize_simd<simd_size_v<T, Abi> * N, simd_mask<T, Abi>>
      concat(const array<simd_mask<T, Abi>, N>& arr) noexcept;
\end{itemdecl}

\begin{itemdescr}
  \pnum\returns
  A data-parallel object, the $i^\text{th}$ element of which is initialized by \tcode{arr[$i$ / simd_size_v<T, Abi>][$i$ \% simd_size_v<T, Abi>]}.
\end{itemdescr}

\rSec2[parallel.simd.alg]{Algorithms}

\begin{itemdecl}
template<class T, class Abi> simd<T, Abi> min(const simd<T, Abi>& a, const simd<T, Abi>& b) noexcept;
\end{itemdecl}

\begin{itemdescr}
  \pnum\returns
  The result of the element-wise application of \tcode{std::min(a[$i$], b[$i$])} \foralli.
\end{itemdescr}

\begin{itemdecl}
template<class T, class Abi> simd<T, Abi> max(const simd<T, Abi>& a, const simd<T, Abi>& b) noexcept;
\end{itemdecl}

\begin{itemdescr}
  \pnum\returns
  The result of the element-wise application of \tcode{std::max(a[$i$], b[$i$])} \foralli.
\end{itemdescr}

\begin{itemdecl}
template<class T, class Abi>
  pair<simd<T, Abi>, simd<T, Abi>> minmax(const simd<T, Abi>& a, const simd<T, Abi>& b) noexcept;
\end{itemdecl}

\begin{itemdescr}
  \pnum\returns
  A \tcode{pair} initialized with
  \begin{itemize}
    \item the result of element-wise application of \tcode{std::min(a[$i$], b[$i$])} \foralli{} in the \tcode{first} member, and
    \item the result of element-wise application of \tcode{std::max(a[$i$], b[$i$])} \foralli{} in the \tcode{second} member.
  \end{itemize}
\end{itemdescr}

\begin{itemdecl}
template<class T, class Abi> simd<T, Abi>
  clamp(const simd<T, Abi>& v, const simd<T, Abi>& lo, const simd<T, Abi>& hi);
\end{itemdecl}

\begin{itemdescr}
  \pnum\requires
  No element in \tcode{lo} shall be greater than the corresponding element in \tcode{hi}.

  \pnum\returns
  The result of element-wise application of \tcode{std::clamp(v[$i$], lo[$i$], hi[$i$])} \foralli.
\end{itemdescr}

\rSec2[parallel.simd.math]{Math library}

\pnum
For each set of overloaded functions within \tcode{<cmath>}, there shall be additional overloads sufficient to ensure that if any argument corresponding to a \tcode{double} parameter has type \tcode{simd<T, Abi>}, where \tcode{is_floating_point_v<T>} is \tcode{true}, then:
\begin{itemize}
  \item All arguments corresponding to \tcode{double} parameters shall be convertible to \tcode{simd<T, Abi>}.
  \item All arguments corresponding to \tcode{double*} parameters shall be of type \tcode{simd<T, Abi>*}.
  \item All arguments corresponding to parameters of integral type \tcode{U} shall be convertible to \tcode{fixed_size_simd<U, simd_size_v<T, Abi>>}.
  \item All arguments corresponding to \tcode{U*}, where \tcode{U} is integral, shall be of type \tcode{fixed_size_simd<U, simd_size_v<T, Abi>>*}.
  \item If the corresponding return type is \tcode{double}, the return type of the additional overloads is \tcode{simd<T, Abi>}. Otherwise, if the corresponding return type is \tcode{bool}, the return type of the additional overload is \tcode{simd_mask<T, Abi>}. Otherwise, the return type is \tcode{fixed_size_simd<R, simd_size_v<T, Abi>>}, with \tcode{R} denoting the corresponding return type.
\end{itemize}
It is unspecified whether a call to these overloads with arguments that are all convertible to \tcode{simd<T, Abi>} but are not of type \tcode{simd<T, Abi>} is well-formed.

\pnum
Each function overload produced by the above rules applies the indicated \tcode{<cmath>} function element-wise. For the mathematical functions, the results per element only need to be approximately equal to the application of the function which is overloaded for the element type.

\pnum
The behavior is undefined if a domain, pole, or range error occurs when the input argument(s) are applied to the indicated \tcode{<cmath>} function.

\pnum
If \tcode{abs} is called with an argument of type \tcode{simd<X, Abi>} for which \tcode{is_unsigned_v<X>} is \tcode{true}, the program is ill-formed.

\rSec1[parallel.simd.mask.class]{Class template \tcode{simd_mask}}

\rSec2[parallel.simd.mask.overview]{Class template \tcode{simd_mask} overview}

\begin{codeblock}
template<class T, class Abi> class simd_mask {
public:
  using value_type = bool;
  using reference = @\seebelow@;
  using simd_type = simd<T, Abi>;
  using abi_type = Abi;

  static constexpr size_t size() noexcept;

  simd_mask() noexcept = default;

  // \ref{parallel.simd.mask.ctor}, Constructors
  explicit simd_mask(value_type) noexcept;
  template<class U>
    simd_mask(const simd_mask<U, simd_abi::fixed_size<size()>>&) noexcept;
  template<class Flags> simd_mask(const value_type* mem, Flags);

  // \ref{parallel.simd.mask.copy}, Copy functions
  template<class Flags> void copy_from(const value_type* mem, Flags);
  template<class Flags> void copy_to(value_type* mem, Flags);

  // \ref{parallel.simd.mask.subscr}, Subscript operators
  reference operator[](size_t);
  value_type operator[](size_t) const;

  // \ref{parallel.simd.mask.unary}, Unary operators
  simd_mask operator!() const noexcept;

  // \ref{parallel.simd.mask.binary}, Binary operators
  friend simd_mask operator&&(const simd_mask&, const simd_mask&) noexcept;
  friend simd_mask operator||(const simd_mask&, const simd_mask&) noexcept;
  friend simd_mask operator&(const simd_mask&, const simd_mask&) noexcept;
  friend simd_mask operator|(const simd_mask&, const simd_mask&) noexcept;
  friend simd_mask operator^(const simd_mask&, const simd_mask&) noexcept;

  // \ref{parallel.simd.mask.cassign}, Compound assignment
  friend simd_mask& operator&=(simd_mask&, const simd_mask&) noexcept;
  friend simd_mask& operator|=(simd_mask&, const simd_mask&) noexcept;
  friend simd_mask& operator^=(simd_mask&, const simd_mask&) noexcept;

  // \ref{parallel.simd.mask.comparison}, Comparisons
  friend simd_mask operator==(const simd_mask&, const simd_mask&) noexcept;
  friend simd_mask operator!=(const simd_mask&, const simd_mask&) noexcept;
};
\end{codeblock}

\pnum
The class template \tcode{simd_mask} is a data-parallel type with the element type \tcode{bool}. The width of a given \tcode{simd_mask} specialization is a constant expression, determined by the template parameters. Specifically, \tcode{simd_mask<T, Abi>::size() == simd<T, Abi>::size()}.

\pnum
Every specialization of \tcode{simd_mask} shall be a complete type. The specialization \tcode{simd_mask<T, Abi>} is supported if \tcode{T} is a vectorizable type and
\begin{itemize}
  \item \tcode{Abi} is \tcode{simd_abi::scalar}, or
  \item \tcode{Abi} is \tcode{simd_abi::fixed_size<N>}, with \tcode{N} constrained as defined in (\ref{parallel.simd.abi}).
\end{itemize}

If \tcode{Abi} is an extended ABI tag, it is implementation-defined whether \tcode{simd_mask<T, Abi>} is supported. \begin{note}The intent is for implementations to decide on the basis of the currently targeted system.\end{note}

If \tcode{simd_mask<T, Abi>} is not supported, the specialization shall have a deleted default constructor, deleted destructor, deleted copy constructor, and deleted copy assignment.
Otherwise, the following are true:
\begin{itemize}
  \item \tcode{is_nothrow_move_constructible_v<simd_mask<T, Abi>>}, and
  \item \tcode{is_nothrow_move_assignable_v<simd_mask<T, Abi>>}, and
  \item \tcode{is_nothrow_default_constructible_v<simd_mask<T, Abi>>}.
\end{itemize}

\pnum
Default initialization performs no intialization of the elements; value-initialization initializes each element with \tcode{false}. \begin{note}Thus, default initialization leaves the elements in an indeterminate state.\end{note}

\pnum
Implementations should enable explicit conversion from and to implementation-defined types. This adds one or more of the following declarations to class \tcode{simd_mask}:

\begin{codeblock}
explicit operator @\impdefx{type(s) \tcode{simd_mask} is explicit convertible to}@() const;
explicit simd_mask(const @\impdefx{type(s) \tcode{simd_mask} is explicit convertible from}@& init) const;
\end{codeblock}

\pnum
The member type \tcode{reference} has the same interface as \tcode{simd<T, Abi>::reference}, except its \tcode{value_type} is \tcode{bool}. (\ref{parallel.simd.reference})

\rSec2[parallel.simd.mask.width]{\tcode{simd_mask} width}

\begin{itemdecl}
static constexpr size_t size() noexcept;
\end{itemdecl}

\begin{itemdescr}
  \pnum\returns
  The width of \tcode{simd<T, Abi>}.
\end{itemdescr}

\rSec2[parallel.simd.mask.ctor]{Constructors}

\begin{itemdecl}
explicit simd_mask(value_type x) noexcept;
\end{itemdecl}

\begin{itemdescr}
  \pnum\effects
  Constructs an object with each element initialized to \tcode{x}.
\end{itemdescr}

\begin{itemdecl}
template<class U> simd_mask(const simd_mask<U, simd_abi::fixed_size<size()>>& x) noexcept;
\end{itemdecl}

\begin{itemdescr}
  \pnum\effects
  Constructs an object of type \tcode{simd_mask} where the $i^\text{th}$ element equals \tcode{x[$i$]} \foralli.

  \pnum\remarks
  This constructor shall not participate in overload resolution unless \tcode{abi_type} is \tcode{simd_abi::fixed_size<size()>}.
\end{itemdescr}

\begin{itemdecl}
template<class Flags> simd_mask(const value_type* mem, Flags);
\end{itemdecl}

\begin{itemdescr}
  \pnum\requires
  \begin{itemize}
    \item \tcode{[mem, mem + size())} is a valid range.
    \flagsRequires{simd_mask}{value_type}
  \end{itemize}

  \pnum\effects
  Constructs an object where the $i^\text{th}$ element is initialized to \tcode{mem[$i$]} \foralli.

  \pnum\throws Nothing.

  \pnum\remarks
  This constructor shall not participate in overload resolution unless \tcode{is_simd_flag_type_v<Flags>} is \tcode{true}.

\end{itemdescr}

\rSec2[parallel.simd.mask.copy]{Copy functions}

\begin{itemdecl}
template<class Flags> void copy_from(const value_type* mem, Flags);
\end{itemdecl}

\begin{itemdescr}
  \pnum\requires
  \begin{itemize}
    \item \tcode{[mem, mem + size())} is a valid range.
    \flagsRequires{simd_mask}{value_type}
  \end{itemize}

  \pnum\effects
  Replaces the elements of the \tcode{simd_mask} object such that the $i^\text{th}$ element is replaced with \tcode{mem[$i$]} \foralli.

  \pnum\throws Nothing.

  \pnum\remarks
  This function shall not participate in overload resolution unless \tcode{is_simd_flag_type_v<Flags>} is \tcode{true}.

\end{itemdescr}

\begin{itemdecl}
template<class Flags> void copy_to(value_type* mem, Flags);
\end{itemdecl}

\begin{itemdescr}
  \pnum\requires
  \begin{itemize}
    \item \tcode{[mem, mem + size())} is a valid range.
    \flagsRequires{simd_mask}{value_type}
  \end{itemize}

  \pnum\effects
  Copies all \tcode{simd_mask} elements as if \tcode{mem[$i$] = operator[]($i$)} \foralli.

  \pnum\throws Nothing.

  \pnum\remarks
  This function shall not participate in overload resolution unless \tcode{is_simd_flag_type_v<Flags>} is \tcode{true}.
\end{itemdescr}

\rSec2[parallel.simd.mask.subscr]{Subscript operators}

\begin{itemdecl}
reference operator[](size_t i);
\end{itemdecl}

\begin{itemdescr}
  \pnum\requires
  \tcode{i < size()}.

  \pnum\returns
  A \tcode{reference} (see \ref{parallel.simd.reference}) referring to the $i^\text{th}$ element.

  \pnum\throws Nothing.
\end{itemdescr}

\begin{itemdecl}
value_type operator[](size_t i) const;
\end{itemdecl}

\begin{itemdescr}
  \pnum\requires
  \tcode{i < size()}.

  \pnum\returns
  The value of the $i^\text{th}$ element.

  \pnum\throws Nothing.
\end{itemdescr}

\rSec2[parallel.simd.mask.unary]{Unary operators}

\begin{itemdecl}
simd_mask operator!() const noexcept;
\end{itemdecl}

\begin{itemdescr}
  \pnum\returns
  The result of the element-wise appliation of \tcode{operator!}.
\end{itemdescr}

\rSec1[parallel.simd.mask.nonmembers]{Non-member operations}

\rSec2[parallel.simd.mask.binary]{Binary operators}

\begin{itemdecl}
friend simd_mask operator&&(const simd_mask& lhs, const simd_mask& rhs) noexcept;
friend simd_mask operator||(const simd_mask& lhs, const simd_mask& rhs) noexcept;
friend simd_mask operator& (const simd_mask& lhs, const simd_mask& rhs) noexcept;
friend simd_mask operator| (const simd_mask& lhs, const simd_mask& rhs) noexcept;
friend simd_mask operator^ (const simd_mask& lhs, const simd_mask& rhs) noexcept;
\end{itemdecl}

\begin{itemdescr}
  \pnum\returns
  A \tcode{simd_mask} object initialized with the results of applying the indicated operator to \tcode{lhs} and \tcode{rhs} as a binary element-wise operation.
\end{itemdescr}

\rSec2[parallel.simd.mask.cassign]{Compound assignment}

\begin{itemdecl}
friend simd_mask& operator&=(simd_mask& lhs, const simd_mask& rhs) noexcept;
friend simd_mask& operator|=(simd_mask& lhs, const simd_mask& rhs) noexcept;
friend simd_mask& operator^=(simd_mask& lhs, const simd_mask& rhs) noexcept;
\end{itemdecl}

\begin{itemdescr}
  \pnum\effects
  These operators apply the indicated operator to \tcode{lhs} and \tcode{rhs} as a binary element-wise operation.

  \pnum\returns
  \tcode{lhs}.
\end{itemdescr}

\rSec2[parallel.simd.mask.comparison]{Comparisons}

\begin{itemdecl}
friend simd_mask operator==(const simd_mask&, const simd_mask&) noexcept;
friend simd_mask operator!=(const simd_mask&, const simd_mask&) noexcept;
\end{itemdecl}

\begin{itemdescr}
  \pnum\returns
  A \tcode{simd_mask} object initialized with the results of applying the indicated operator to \tcode{lhs} and \tcode{rhs} as a binary element-wise operation.
\end{itemdescr}

\rSec2[parallel.simd.mask.reductions]{Reductions}

\begin{itemdecl}
template<class T, class Abi> bool all_of(const simd_mask<T, Abi>& k) noexcept;
\end{itemdecl}

\begin{itemdescr}
  \pnum\returns
  \tcode{true} if all boolean elements in \tcode{k} are \tcode{true}, \tcode{false} otherwise.
\end{itemdescr}

\begin{itemdecl}
template<class T, class Abi> bool any_of(const simd_mask<T, Abi>& k) noexcept;
\end{itemdecl}

\begin{itemdescr}
  \pnum\returns
  \tcode{true} if at least one boolean element in \tcode{k} is \tcode{true}, \tcode{false} otherwise.
\end{itemdescr}

\begin{itemdecl}
template<class T, class Abi> bool none_of(const simd_mask<T, Abi>& k) noexcept;
\end{itemdecl}

\begin{itemdescr}
  \pnum\returns
  \tcode{true} if none of the one boolean elements in \tcode{k} is \tcode{true}, \tcode{false} otherwise.
\end{itemdescr}

\begin{itemdecl}
template<class T, class Abi> bool some_of(const simd_mask<T, Abi>& k) noexcept;
\end{itemdecl}

\begin{itemdescr}
  \pnum\returns
  \tcode{true} if at least one of the one boolean elements in \tcode{k} is \tcode{true} and at least one of the boolean elements in \tcode{k} is \tcode{false}, \tcode{false} otherwise.
\end{itemdescr}

\begin{itemdecl}
template<class T, class Abi> int popcount(const simd_mask<T, Abi>& k) noexcept;
\end{itemdecl}

\begin{itemdescr}
  \pnum\returns
  The number of boolean elements in \tcode{k} that are \tcode{true}.
\end{itemdescr}

\begin{itemdecl}
template<class T, class Abi> int find_first_set(const simd_mask<T, Abi>& k);
\end{itemdecl}

\begin{itemdescr}
  \pnum\requires
  \tcode{any_of(k)} returns \tcode{true}.

  \pnum\returns
  The lowest element index $i$ where \tcode{k[$i$]} is \tcode{true}.

  \pnum\throws Nothing.
\end{itemdescr}

\begin{itemdecl}
template<class T, class Abi> int find_last_set(const simd_mask<T, Abi>& k);
\end{itemdecl}

\begin{itemdescr}
  \pnum\requires
  \tcode{any_of(k)} returns \tcode{true}.

  \pnum\returns
  The greatest element index $i$ where \tcode{k[$i$]} is \tcode{true}.

  \pnum\throws Nothing.
\end{itemdescr}

\begin{itemdecl}
bool all_of(@\UNSP{T}@) noexcept;
bool any_of(@\UNSP{T}@) noexcept;
bool none_of(@\UNSP{T}@) noexcept;
bool some_of(@\UNSP{T}@) noexcept;
int popcount(@\UNSP{T}@) noexcept;
\end{itemdecl}

\begin{itemdescr}
  \pnum\returns
  \tcode{all_of} and \tcode{any_of} return their arguments; \tcode{none_of} returns the negation of its argument; \tcode{some_of} returns \tcode{false}; \tcode{popcount} returns the integral representation of its argument.

  \pnum\remarks
  The parameter type \tcode{T} is an unspecified type that is only constructible via implicit conversion from \tcode{bool}.
\end{itemdescr}

\begin{itemdecl}
int find_first_set(@\UNSP{T}@);
int find_last_set(@\UNSP{T}@);
\end{itemdecl}

\begin{itemdescr}
  \pnum\requires
  The value of the argument is \tcode{true}.

  \pnum\returns
  \tcode{0}.

  \pnum\throws Nothing.

  \pnum\remarks
  The parameter type \tcode{T} is an unspecified type that is only constructible via implicit conversion from \tcode{bool}.
\end{itemdescr}

\rSec2[parallel.simd.mask.where]{\tcode{where} functions}

\begin{itemdecl}
template<class T, class Abi>
  where_expression<simd_mask<T, Abi>, simd<T, Abi>>
    where(const typename simd<T, Abi>::mask_type& k, simd<T, Abi>& v) noexcept;
template<class T, class Abi>
  const_where_expression<simd_mask<T, Abi>, simd<T, Abi>>
    where(const typename simd<T, Abi>::mask_type& k, const simd<T, Abi>& v) noexcept;
template<class T, class Abi>
  where_expression<simd_mask<T, Abi>, simd_mask<T, Abi>>
    where(const type_identity_t<simd_mask<T, Abi>>& k, simd_mask<T, Abi>& v) noexcept;
template<class T, class Abi>
  const_where_expression<simd_mask<T, Abi>, simd_mask<T, Abi>>
    where(const type_identity_t<simd_mask<T, Abi>>& k, const simd_mask<T, Abi>& v) noexcept;
\end{itemdecl}

\begin{itemdescr}
  \pnum\returns
  An object (\ref{parallel.simd.whereexpr}) with \tcode{mask} and \tcode{data} initialized with \tcode{k} and \tcode{v} respectively.
\end{itemdescr}

\begin{itemdecl}
template<class T>
  where_expression<bool T>
    where(@\seebelow@ k, T& v) noexcept;
template<class T>
  const_where_expression<bool, T>
    where(@\seebelow@ k, const T& v) noexcept;
\end{itemdecl}

\begin{itemdescr}
  \pnum\remarks
  The functions shall not participate in overload resolution unless

  \begin{itemize}
    \item \tcode{T} is neither a \tcode{simd} nor a \tcode{simd_mask} specialization, and
    \item the first argument is of type \tcode{bool}.
  \end{itemize}

  \pnum\returns
  An object (\ref{parallel.simd.whereexpr}) with \tcode{mask} and \tcode{data} initialized with \tcode{k} and \tcode{v} respectively.
\end{itemdescr}
