%!TEX root = std.tex
\chapter{Foreword}

%ISO (the International Organization for Standardization)
%is a worldwide federation of national standards bodies (ISO member bodies).
%The work of preparing International Standards
%is normally carried out through ISO technical committees.
%Each member body interested in a subject
%for which a technical committee has been established
%has the right to be represented on that committee.
%International organizations,
%governmental and non-governmental,
%in liaison with ISO,
%also take part in the work.
%ISO collaborates closely with
%the International Electrotechnical Commission (IEC)
%on all matters of electrotechnical standardization.

ISO (the International Organization for Standardization) and IEC (the
International Electrotechnical Commission) form the specialized system for
worldwide standardization. National bodies that are members of ISO or IEC
participate in the development of International Standards through technical
committees established by the respective organization to deal with particular
fields of technical activity. ISO and IEC technical committees collaborate in
fields of mutual interest. Other international organizations, governmental and
non-governmental, in liaison with ISO and IEC, also take part in the work. In
the field of information technology, ISO and IEC have established a joint
technical committee, ISO/IEC JTC 1.

The procedures used to develop this document and those intended for its further
maintenance are described in the ISO/IEC Directives, Part 1. In particular the
different approval criteria needed for the different types of ISO documents should
be noted. This document was drafted in accordance with the editorial rules of
the ISO/IEC Directives, Part 2
(see \href{http://www.iso.org/directives}{\tcode{www.iso.org/directives}}).

Attention is drawn to the possibility that some of the elements of this
document may be the subject of patent rights. ISO shall not be held
responsible for identifying any or all such patent rights. Details of any
patent rights identified during the development of the document will be in the
Introduction and/or on the ISO list of patent declarations received
(see \href{http://www.iso.org/patents}{\tcode{www.iso.org/patents}}).

Any trade name used in this document is information given for the convenience
of users and does not constitute an endorsement.

For an explanation on
the voluntary nature of standards,
the meaning of ISO specific terms and expressions related
to conformity assessment, as well as information about ISO's adherence
to the World Trade Organization (WTO) principles
in the Technical Barriers to Trade (TBT) see the following URL:
\href{http://www.iso.org/iso/foreword.html}{\tcode{www.iso.org/iso/foreword.html}}.

This document was prepared by
Technical Committee ISO/IEC JTC 1, \textit{Information technology},
Subcommittee SC 22, \textit{Programming languages, their environments and system software interfaces}.
