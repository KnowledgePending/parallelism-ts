%!TEX root = ts.tex

\rSec0[parallel.task_block]{Task Block}

\rSec1[parallel.task_block.synopsis]{Header \tcode{<experimental/task_block> synopsis}}

\begin{codeblock}
namespace std::experimental {
inline namespace parallelism_v2 {
  class task_cancelled_exception;

  class task_block;

  template<class F>
    void define_task_block(F&& f);

  template<class f>
    void define_task_block_restore_thread(F&& f);
}
}
\end{codeblock}

\rSec1[parallel.task_block.task_cancelled_exception]{Class \tcode{task_cancelled_exception}}

\begin{codeblock}
namespace std::experimental {
inline namespace parallelism_v2 {

  class task_cancelled_exception : public exception
  {
    public:
      task_cancelled_exception() noexcept;
      virtual const char* what() const noexcept override;
  };
}
}
\end{codeblock}

\pnum The class \tcode{task_cancelled_exception} defines the type of objects
thrown by \tcode{task_block::run} or \tcode{task_block::wait} if they detect
than an exception is pending within the current parallel block. See
\ref{parallel.task_block.exceptions}, below.

\rSec2[parallel.task_block.task_cancelled_exception.what]{\tcode{task_cancelled_exception} member function \tcode{what}}

\begin{itemdecl}
virtual const char* what() const noexcept
\end{itemdecl}

\begin{itemdescr}
\pnum \returns An implementation-defined NTBS.
\end{itemdescr}

\rSec1[parallel.task_block.class]{Class \tcode{task_block}}

\begin{codeblock}
namespace std::experimental {
inline namespace parallelism_v2 {

  class task_block
  {
    private:
      ~task_block();

    public:
      task_block(const task_block&) = delete;
      task_block& operator=(const task_block&) = delete;
      void operator&() const = delete;

      template<class F>
        void run(F&& f);

      void wait();
  };
}
}
\end{codeblock}

\begin{itemdescr}
\pnum The class \tcode{task_block} defines an interface for forking and joining parallel tasks. The \tcode{define_task_block} and \tcode{define_task_block_restore_thread} function templates create an object of type \tcode{task_block} and pass a reference to that object to a user-provided function object.

\pnum An object of class \tcode{task_block} cannot be constructed, destroyed,
copied, or moved except by the implementation of the task block library.
Taking the address of a \tcode{task_block} object via \tcode{operator\&} is
ill-formed. Obtaining its address by any other means (including
    \tcode{addressof}) results in a pointer with an unspecified value;
dereferencing such a pointer results in undefined behavior.

\pnum A \tcode{task_block} is \defn{active} if it was created by the \defn{nearest
  enclosing} \defn{task block}, where \defn{task block} refers to an invocation
  of \tcode{define_task_block} or \tcode{define_task_block_restore_thread} and
  \defn{nearest enclosing} means the most recent invocation that has not yet
  completed. Code designated for execution in another thread by means other
  than the facilities in this section (e.g., using \tcode{thread} or
      \tcode{async}) are not enclosed in the task block and a
  \tcode{task_block} passed to (or captured by) such code is not active within
  that code. Performing any operation on a \tcode{task_block} that is not
  active results in undefined behavior.

\pnum When the argument to \tcode{task_block::run} is called, no
\tcode{task_block} is active, not even the \tcode{task_block} on which
\tcode{run} was called. (The function object should not, therefore, capture a
    \tcode{task_block} from the surrounding block.)

\begin{example}
\begin{codeblock}
define_task_block([&](auto& tb) {
  tb.run([&]{
    tb.run([] { f(); });               // Error: tb is not active within run
    define_task_block([&](auto& tb2) { // Define new task block
      tb2.run(f);
      ...
    });
  });
  ...
});
\end{codeblock}
\end{example}

\begin{note}
Implementations are encouraged to diagnose the above error at translation time.
\end{note}

\end{itemdescr}

