%!TEX root = ts.tex

\rSec0[parallel.task_block]{Task Block}

\rSec1[parallel.task_block.synopsis]{Header \tcode{<experimental/task_block> synopsis}}

\begin{codeblock}
namespace std::experimental {
inline namespace parallelism_v2 {
  class task_cancelled_exception;

  class task_block;

  template<class F>
    void define_task_block(F&& f);

  template<class f>
    void define_task_block_restore_thread(F&& f);
}
}
\end{codeblock}

\rSec1[parallel.task_block.task_cancelled_exception]{Class \tcode{task_cancelled_exception}}

\begin{codeblock}
namespace std::experimental {
inline namespace parallelism_v2 {

  class task_cancelled_exception : public exception
  {
  public:
    task_cancelled_exception() noexcept;
    virtual const char* what() const noexcept override;
  };
}
}
\end{codeblock}

\pnum
The class \tcode{task_cancelled_exception} defines the type of objects
thrown by \tcode{task_block\colcol{}run} or \tcode{task_block\colcol{}wait} if they detect
than an exception is pending within the current parallel block. See
\ref{parallel.task_block.exceptions}, below.

\rSec2[parallel.task_block.task_cancelled_exception.what]{\tcode{task_cancelled_exception} member function \tcode{what}}

\begin{itemdecl}
virtual const char* what() const noexcept;
\end{itemdecl}

\begin{itemdescr}
\pnum
\returns An implementation-defined NTBS.
\end{itemdescr}

\rSec1[parallel.task_block.class]{Class \tcode{task_block}}

\begin{codeblock}
namespace std::experimental {
inline namespace parallelism_v2 {

  class task_block
  {
  private:
    ~task_block();

  public:
    task_block(const task_block&) = delete;
    task_block& operator=(const task_block&) = delete;
    void operator&() const = delete;

    template<class F>
      void run(F&& f);

    void wait();
  };
}
}
\end{codeblock}

\begin{itemdescr}
\pnum
The class \tcode{task_block} defines an interface for forking and joining parallel tasks. The \tcode{define_task_block} and \tcode{define_task_block_restore_thread} function templates create an object of type \tcode{task_block} and pass a reference to that object to a user-provided function object.

\pnum
An object of class \tcode{task_block} cannot be constructed, destroyed,
copied, or moved except by the implementation of the task block library.
Taking the address of a \tcode{task_block} object via \tcode{operator\&} is
ill-formed. Obtaining its address by any other means (including
    \tcode{addressof}) results in a pointer with an unspecified value;
dereferencing such a pointer results in undefined behavior.

\pnum
A \tcode{task_block} is \defn{active} if it was created by the \defn{nearest
  enclosing} \defn{task block}, where \defn{task block} refers to an invocation
  of \tcode{define_task_block} or \tcode{define_task_block_restore_thread} and
  \defn{nearest enclosing} means the most recent invocation that has not yet
  completed. Code designated for execution in another thread by means other
  than the facilities in this section (e.g., using \tcode{thread} or
      \tcode{async}) are not enclosed in the task block and a
  \tcode{task_block} passed to (or captured by) such code is not active within
  that code. Performing any operation on a \tcode{task_block} that is not
  active results in undefined behavior.

\pnum
When the argument to \tcode{task_block::run} is called, no
\tcode{task_block} is active, not even the \tcode{task_block} on which
\tcode{run} was called. (The function object should not, therefore, capture a
    \tcode{task_block} from the surrounding block.)

\begin{example}
\begin{codeblock}
define_task_block([&](auto& tb) {
  tb.run([&]{
    tb.run([] { f(); });               // Error: tb is not active within run
    define_task_block([&](auto& tb2) { // Define new task block
      tb2.run(f);
      ...
    });
  });
  ...
});
\end{codeblock}
\end{example}

\begin{note}
Implementations are encouraged to diagnose the above error at translation time.
\end{note}

\end{itemdescr}

\rSec2[parallel.task_block.class.run]{\tcode{task_block} member function template \tcode{run}}

\begin{itemdecl}
template<class F> void run(F&& f);
\end{itemdecl}

\begin{itemdescr}
\pnum
\requires \tcode{F} shall be \tcode{MoveConstructible}. \tcode{DECAY_COPY(std::forward<F>(f))()} shall be a valid expression.

\pnum
\preconditions \tcode{*this} shall be the active \tcode{task_block}.

\pnum
\effects Evaluates \tcode{DECAY_COPY(std::forward<F>(f))()}, where
\tcode{DECAY_COPY(std::forward<F>(f))} is evaluated synchronously within the
current thread. The call to the resulting copy of the function object is
permitted to run on an unspecified thread created by the implementation in an
unordered fashion relative to the sequence of operations following the call to
\tcode{run(f)} (the continuation), or indeterminately sequenced within the same thread
as the continuation. The call to \tcode{run} synchronizes with the call to the function
object. The completion of the call to the function object synchronizes with the
next invocation of wait on the same \tcode{task_block} or completion of the nearest
enclosing task block (i.e., the \tcode{define_task_block} or
    \tcode{define_task_block_restore_thread} that created this \tcode{task_block}).

\pnum
\throws \tcode{task_cancelled_exception}, as described in \ref{parallel.task_block.exceptions}.

\pnum
\remarks The \tcode{run} function may return on a thread other than the
one on which it was called; in such cases, completion of the call to
\tcode{run} synchronizes with the continuation. \begin{note}The return from
\tcode{run} is ordered similarly to an ordinary function call in a single
thread.\end{note}

\pnum
\remarks The invocation of the user-supplied function object \tcode{f}
may be immediate or may be delayed until compute resources are available.
\tcode{run} might or might not return before the invocation of \tcode{f}
completes.

\end{itemdescr}

\rSec2[parallel.task_block.class.wait]{\tcode{task_block} member function \tcode{wait}}

\begin{itemdecl}
void wait();
\end{itemdecl}

\begin{itemdescr}
\pnum
\realpreconditions \tcode{*this} shall be the active \tcode{task_block}.

\pnum
\effects Blocks until the tasks spawned using this \tcode{task_block} have completed.

\pnum
\throws \tcode{task_cancelled_exception}, as described in \ref{parallel.task_block.exceptions}.

\pnum
\postconditions All tasks spawned by the nearest enclosing task block have completed.

\pnum
\remarks The \tcode{wait} function may return on a thread other than the
one on which it was called; in such cases, completion of the call to
\tcode{wait} synchronizes with subsequent operations. \begin{note}The return
from \tcode{wait} is ordered similarly to an ordinary function call in a single
thread.\end{note}

\begin{example}
\begin{codeblock}
define_task_block([&](auto& tb) {
  tb.run([&]{ process(a, w, x); });    // Process a[w] through a[x]
  if (y < x) tb.wait();                // Wait if overlap between [w,x) and [y,z)
  process(a, y, z);                    // Process a[y] through a[z]
});
\end{codeblock}
\end{example}

\end{itemdescr}

\rSec1[parallel.task_block.define_task_block]{Function template \tcode{define_task_block}}

\begin{itemdecl}
template<class F> void define_task_block(F&& f);
template<class F> void define_task_block_restore_thread(F&& f);
\end{itemdecl}

\begin{itemdescr}
\pnum
\requires Given an lvalue \tcode{tb} of type \tcode{task_block}, the expression \tcode{f(tb)} shall be well-formed

\pnum
\effects Constructs a \tcode{task_block} \tcode{tb} and calls \tcode{f(tb)}.

\pnum
\throws \tcode{exception_list}, as specified in \ref{parallel.task_block.exceptions}.

\pnum
\postconditions All tasks spawned from \tcode{f} have finished execution.

\pnum
\remarks The \tcode{define_task_block} function may return on a thread
other than the one on which it was called unless there are no task blocks
active on entry to \tcode{define_task_block} (see
    \ref{parallel.task_block.class}), in which case the function returns on the
original thread. When \tcode{define_task_block} returns on a different thread, it
synchronizes with operations following the call. \begin{note}The return from
\tcode{define_task_block} is ordered similarly to an ordinary function
call in a single thread.\end{note} The \tcode{define_task_block_restore_thread}
function always returns on the same thread as the one on which it was called.

\pnum
\realnotes It is expected (but not mandated) that \tcode{f} will (directly or indirectly) call \tcode{tb.run({\em function-\brk{}object})}.
\end{itemdescr}

\rSec1[parallel.task_block.exceptions]{Exception Handling}

\pnum
Every \tcode{task_block} has an associated exception list. When the task block starts, its associated exception list is empty.

\pnum
When an exception is thrown from the user-provided function object passed
to \tcode{define_task_block} or \tcode{define_task_block_restore_thread}, it is
added to the exception list for that task block. Similarly, when an exception
is thrown from the user-provided function object passed into
\tcode{task_block::run}, the exception object is added to the exception list
associated with the nearest enclosing task block. In both cases, an
implementation may discard any pending tasks that have not yet been invoked.
Tasks that are already in progress are not interrupted except at a call to
\tcode{task_block::run} or \tcode{task_block::wait} as described below.

\pnum
If the implementation is able to detect that an exception has been thrown
by another task within the same nearest enclosing task block, then
\tcode{task_block::run} or \tcode{task_block::wait} may throw
\tcode{task_canceled_exception}; these instances of
\tcode{task_canceled_exception} are not added to the exception list of the
corresponding task block.

\pnum
When a task block finishes with a non-empty exception list, the
exceptions are aggregated into an \tcode{exception_list} object, which is then thrown
from the task block.

\pnum
The order of the exceptions in the \tcode{exception_list} object is unspecified.

