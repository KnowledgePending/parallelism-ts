%!TEX root = ts.tex

\rSec0[parallel.exceptions]{Parallel exceptions}

\rSec1[parallel.exceptions.synopsis]{Header \tcode{<experimental/exception_list>} synopsis}

\begin{codeblock}
namespace std::experimental {
inline namespace parallelism_v2 {

  class exception_list : public exception {
  public:
    using iterator = @\unspec@;

    size_t size() const noexcept;
    iterator begin() const noexcept;
    iterator end() const noexcept;

    const char* what() const noexcept override;
  };
}
}
\end{codeblock}

\pnum
The class \tcode{exception_list} owns a sequence of \tcode{exception_ptr} objects.

\pnum
\tcode{exception_list::iterator} is an iterator which meets the forward iterator requirements and has a value type of \tcode{exception_ptr}.

\begin{itemdecl}
size_t size() const noexcept;
\end{itemdecl}

\begin{itemdescr}
  \pnum
  \returns The number of \tcode{exception_ptr} objects contained within the \tcode{exception_list}.

  \pnum
  \complexity Constant time.
\end{itemdescr}

\begin{itemdecl}
iterator begin() const noexcept;
\end{itemdecl}

\begin{itemdescr}
  \pnum
  \returns An iterator referring to the first \tcode{exception_ptr} object returned within the \tcode{exception_list}.
\end{itemdescr}

\begin{itemdecl}
iterator end() const noexcept;
\end{itemdecl}

\begin{itemdescr}
  \pnum
  \returns An iterator that is past the end of the owned sequence.
\end{itemdescr}

\begin{itemdecl}
const char* what() const noexcept override;
\end{itemdecl}

\begin{itemdescr}
  \pnum
  \returns An implementation-defined NTBS.
\end{itemdescr}

