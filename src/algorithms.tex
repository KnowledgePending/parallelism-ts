%!TEX root = ts.tex

\rSec0[parallel.alg]{Parallel algorithms}

\rSec1[parallel.alg.wavefront]{Wavefront Application}

\pnum For the purposes of this section, an \defn{evaluation} is a value
computation or side effect of an expression, or an execution of a statement.
Initialization of a temporary object is considered a subexpression of the
expression that necessitates the temporary object.

\pnum An evaluation A \defn{contains} an evaluation B if:

\begin{itemize}
\item A and B are not potentially concurrent ([intro.races]); and
\item the start of A is the start of B or the start of A is sequenced before the start of B; and
\item the completion of B is the completion of A or the completion of B is sequenced before the completion of A.
\end{itemize}

\begin{note}
This includes evaluations occurring in function invocations.
\end{note}

\pnum An evaluation A is ordered before an evaluation B if A is
deterministically sequenced before B. \begin{note}If A is indeterminately
sequenced with respect to B or A and B are unsequenced, then A is not ordered
before B and B is not ordered before A. The ordered before relationship is
transitive.\end{note}

\pnum For an evaluation A ordered before an evaluation B, both contained in the
same invocation of an element access function, A is a \defn{vertical antecedent} of B if:

\begin{itemize}
\item there exists an evaluation S such that:
  \begin{itemize}
    \item S contains A, and
    \item S contains all evaluations C (if any) such that A is ordered before C and C is ordered before B,
    \item but S does not contain B, and
  \end{itemize}
\item control reached B from A without executing any of the following:
  \begin{itemize}
    \item a \tcode{goto} statement or \tcode{asm} declaration that jumps to a statement outside of S, or
    \item a \tcode{switch} statement executed within S that transfers control into a substatement of a nested selection or iteration statement, or
    \item a \tcode{throw} \begin{note}Even if caught\end{note}, or
    \item a \tcode{jongjmp}.
  \end{itemize}
\end{itemize}

\begin{note}
Vertical antecedent is an irreflexive, antisymmetric, nontransitive relationship between two evaluations. Informally, A is a vertical antecedent of B if A is sequenced immediately before B or A is nested zero or more levels within a statement S that immediately precedes B.
\end{note}

\pnum In the following, $X_i$ and $X_j$ refer to evaluations of the {\em same} expression
or statement contained in the application of an element access function
corresponding to the i\textsuperscript{th} and j\textsuperscript{th} elements of the input sequence.
\begin{note}There might be several evaluations $X_k$, $Y_k$, etc. of a single
expression or statement in application $k$, for example, if the expression or
statement appears in a loop within the element access function.\end{note}

\pnum \defn{Horizontally matched} is an equivalence relationship between two
evaluations of the same expression. An evaluation B\textsubscript{i} is
\defn{horizontally matched} with an evaluation B\textsubscript{j} if:

\begin{itemize}
\item both are the first evaluations in their respective applications of the element access function, or
\item there exist horizontally matched evaluations A\textsubscript{i} and A\textsubscript{j} that are vertical antecedents of evaluations B\textsubscript{i} and B\textsubscript{j}, respectively.
\end{itemize}

\begin{note}\defn{Horizontally matched} establishes a theoretical {\em lock-step} relationship between evaluations in different applications of an element access function.\end{note}

\pnum Let $f$ be a function called for each argument list in a sequence of argument lists. \defn{Wavefront application of f} requires that evaluation A\textsubscript{i} be sequenced before evaluation B\textsubscript{j} if i < j and:

\begin{itemize}
\item A\textsubscript{i} is sequenced before some evaluation B\textsubscript{i} and B\textsubscript{i} is horizontally matched with B\textsubscript{j}, or
\item A\textsubscript{i} is horizontally matched with some evaluation A\textsubscript{j} and A\textsubscript{j} is sequenced before B\textsubscript{j}.
\end{itemize}

\begin{note}
\defn{Wavefront application} guarantees that parallel applications i and j execute such that progress on application j never gets {\em ahead} of application i.
\end{note}
\begin{note}
The relationships between A\textsubscript{i} and B\textsubscript{i} and between A\textsubscript{j} and B\textsubscript{j} are \defn{sequenced before}, not \defn{vertical antecedent}.
\end{note}

\rSec1[parallel.alg.ops]{Non-Numeric Parallel Algorithms}

\rSec2[parallel.alg.ops.synopsis]{Header \tcode{<experimental/algorithm>} synopsis}

\begin{codeblock}
#include <algorithm>

namespace std::experimental {
inline namespace parallelism_v2 {
namespace execution {
  // \ref{parallel.alg.novec}, No vec
  template<class F>
    auto no_vec(F&& f) noexcept -> decltype(std::forward<F>(f)());

  // \ref{parallel.alg.ordupdate.class}, Ordered update class
  template<class T>
    class ordered_update_t;

  // \ref{parallel.alg.ordupdate.func}, Ordered update function template
  template<class T>
    ordered_update_t<T> ordered_update(T& ref) noexcept;
}

// Exposition only: Suppress template argument deduction.
template<class T> struct no_deduce { using type = T; };
template<class T> using no_deduce_t = typename no_deduce<T>::type;

// \ref{parallel.alg.reductions}, Support for reductions
template<class T, class BinaryOperation>
  unspecified reduction(T& var, const T& identity, BinaryOperation combiner);
template<class T>
  unspecified reduction_plus(T& var);
template<class T>
  unspecified reduction_multiplies(T& var);
template<class T>
  unspecified reduction_bit_and(T& var);
template<class T>
  unspecified reduction_bit_or(T& var);
template<class T>
  unspecified reduction_bit_xor(T& var);
template<class T>
  unspecified reduction_min(T& var);
template<class T>
  unspecified reduction_max(T& var);

// \ref{parallel.alg.inductions}, Support for inductions
template<class T>
  unspecified induction(T&& var);
template<class T, class S>
  unspecified induction(T&& var, S stride);

// \ref{parallel.alg.forloop}, For loop
template<class I, class... Rest>
  void for_loop(no_deduce_t<I> start, I finish, Rest&&... rest);
template<class ExecutionPolicy,
         class I, class... Rest>
  void for_loop(ExecutionPolicy&& exec,
                no_deduce_t<I> start, I finish, Rest&&... rest);
template<class I, class S, class... Rest>
  void for_loop_strided(no_deduce_t<I> start, I finish,
                        S stride, Rest&&... rest);
template<class ExecutionPolicy,
         class I, class S, class... Rest>
  void for_loop_strided(ExecutionPolicy&& exec,
                        no_deduce_t<I> start, I finish,
                        S stride, Rest&&... rest);
template<class I, class Size, class... Rest>
  void for_loop_n(I start, Size n, Rest&&... rest);
template<class ExecutionPolicy,
         class I, class Size, class... Rest>
  void for_loop_n(ExecutionPolicy&& exec,
                  I start, Size n, Rest&&... rest);
template<class I, class Size, class S, class... Rest>
  void for_loop_n_strided(I start, Size n, S stride, Rest&&... rest);
template<class ExecutionPolicy,
         class I, class Size, class S, class... Rest>
  void for_loop_n_strided(ExecutionPolicy&& exec,
                          I start, Size n, S stride, Rest&&... rest);
}
}
\end{codeblock}

\rSec2[parallel.alg.reductions]{Reductions}

\pnum Each of the function templates in this subclause
([parallel.alg.reductions]) returns a \defn{reduction object} of unspecified type
having a \defn{reduction value type} and encapsulating a \defn{reduction identity} value for
the reduction, a \defn{combiner} function object, and a \defn{live-out object} from which the
initial value is obtained and into which the final value is stored.

\pnum An algorithm uses reduction objects by allocating an unspecified number
of instances, known as \defn{accumulators}, of the reduction value type. \begin{note}An
implementation might, for example, allocate an accumulator for each thread in
its private thread pool.\end{note} Each accumulator is initialized with the
object's reduction identity, except that the live-out object (which was
    initialized by the caller) comprises one of the accumulators. The algorithm
passes a reference to an accumulator to each application of an element-access
function, ensuring that no two concurrently executing invocations share the
same accumulator. An accumulator can be shared between two applications that do
not execute concurrently, but initialization is performed only once per
accumulator.

\pnum Modifications to the accumulator by the application of element access
functions accrue as partial results. At some point before the algorithm
returns, the partial results are combined, two at a time, using the reduction
object's combiner operation until a single value remains, which is then
assigned back to the live-out object. \begin{note}In order to produce useful
results, modifications to the accumulator should be limited to commutative
operations closely related to the combiner operation. For example if the
combiner is \tcode{plus<T>}, incrementing the accumulator would be consistent with the
combiner but doubling it or assigning to it would not.\end{note}

\begin{itemdecl}
template<class T, class BinaryOperation>
  @\unspec@ reduction(T& var, const T& identity, BinaryOperation combiner);
\end{itemdecl}

\begin{itemdescr}
  \pnum \requires \tcode{T} shall meet the requirements of \tcode{CopyConstructible} and \tcode{MoveAssignable}. The expression \tcode{var = combiner(var, var)} shall be well-formed.

  \pnum \returns A reduction object of unspecified type having reduction value type \tcode{T}, reduction identity \tcode{identity}, combiner function object \tcode{combiner}, and using the object referenced by \tcode{var} as its live-out object.
\end{itemdescr}

\begin{itemdecl}
template<class T>
  @\unspec@ unspecified reduction_plus(T& var);
template<class T>
  @\unspec@ unspecified reduction_multiplies(T& var);
template<class T>
  @\unspec@ unspecified reduction_bit_and(T& var);
template<class T>
  @\unspec@ unspecified reduction_bit_or(T& var);
template<class T>
  @\unspec@ unspecified reduction_bit_xor(T& var);
template<class T>
  @\unspec@ unspecified reduction_min(T& var);
template<class T>
  @\unspec@ unspecified reduction_max(T& var);
\end{itemdecl}

\begin{itemdescr}
  \pnum \requires \tcode{T} shall meet the requirements of \tcode{CopyConstructible} and \tcode{MoveAssignable}.

  \pnum \returns A reduction object of unspecified type having reduction value type \tcode{T}, reduction identity and combiner operation as specified in Table~\ref{tab:parallel.alg.reductions} and using the object referenced by \tcode{var} as its live-out object.
\end{itemdescr}

\begin{floattable}{Reduction identities and combiner operations}{tab:parallel.alg.reductions}
{ccc}
\topline
\chdr{Function} & \chdr{Reduction Identity} & \chdr{Combiner Operation} \\
\capsep
\tcode{reduction_plus} & \tcode{T()} & \tcode{x + y} \\
\hline
\tcode{reduction_multiplies} & \tcode{T(1)} & \tcode{x * y} \\
\hline
\tcode{reduction_bit_and} & \tcode{(\~T())} & \tcode{x \& y} \\
\hline
\tcode{reduction_bit_or} & \tcode{T()} & \tcode{x | y} \\
\hline
\tcode{reduction_bit_xor} & \tcode{T()} & \tcode{x \^{} y} \\
\hline
\tcode{reduction_min} & \tcode{var} & \tcode{min(x, y)} \\
\hline
\tcode{reduction_max} & \tcode{var} & \tcode{max(x, y)} \\
\end{floattable}

\begin{example}
The following code updates each element of \tcode{y} and sets \tcode{s} to the sum of the squares.
\begin{codeblock}
extern int n;
extern float x[], y[], a;
float s = 0;
for_loop(execution::vec, 0, n,
    reduction(s, 0.0f, plus<>()),
    [&](int i, float& accum) {
            y[i] += a*x[i];
            accum += y[i]*y[i];
    }
);
\end{codeblock}
\end{example}

\rSec2[parallel.alg.inductions]{Inductions}

\pnum Each of the function templates in this section return an \defn{induction object} of unspecified type having an \defn{induction value type} and encapsulating an initial value \defn{i} of that type and, optionally, a \defn{stride}.

\pnum For each element in the input range, an algorithm over input sequence S computes an \defn{induction value} from an induction variable and ordinal position \defn{p} within S by the formula \defn{i + p * stride} if a stride was specified or \defn{i + p} otherwise. This induction value is passed to the element access function.

\pnum An induction object may refer to a \defn{live-out} object to hold the final value of the induction sequence. When the algorithm using the induction object completes, the live-out object is assigned the value \defn{i + n * stride}, where \defn{n} is the number of elements in the input range.

\begin{itemdecl}
template<class T>
  @\unspec@ induction(T&& var);
template<class T, class S>
  @\unspec@ induction(T&& var, S stride);
\end{itemdecl}

\begin{itemdescr}
\pnum \returns An induction object with induction value type \tcode{remove_cv_t<remove_reference_t<T>>}, initial value \tcode{var}, and (if specified) stride \tcode{stride}. If \tcode{T} is an lvalue reference to non-\tcode{const} type, then the object referenced by \tcode{var} becomes the live-out object for the induction object; otherwise there is no live-out object.
\end{itemdescr}

\rSec2[parallel.alg.forloop]{For loop}

\begin{itemdecl}
template<class I, class... Rest>
void for_loop(no_deduce_t<I> start, I finish, Rest&&... rest);
template<class ExecutionPolicy,
      class I, class... Rest>
void for_loop(ExecutionPolicy&& exec,
              no_deduce_t<I> start, I finish, Rest&&... rest);

template<class I, class S, class... Rest>
void for_loop_strided(no_deduce_t<I> start, I finish,
                      S stride, Rest&&... rest);
template<class ExecutionPolicy,
      class I, class S, class... Rest>
void for_loop_strided(ExecutionPolicy&& exec,
                      no_deduce_t<I> start, I finish,
                      S stride, Rest&&... rest);

template<class I, class Size, class... Rest>
void for_loop_n(I start, Size n, Rest&&... rest);
template<class ExecutionPolicy,
      class I, class Size, class... Rest>
void for_loop_n(ExecutionPolicy&& exec,
                I start, Size n, Rest&&... rest);
          
template<class I, class Size, class S, class... Rest>
void for_loop_n_strided(I start, Size n, S stride, Rest&&... rest);
template<class ExecutionPolicy, 
      class I, class Size, class S, class... Rest>
void for_loop_n_strided(ExecutionPolicy&& exec,
                        I start, Size n, S stride, Rest&&... rest);
\end{itemdecl}

\begin{itemdescr}
\pnum \requires For the overloads with an \tcode{ExecutionPolicy}, \tcode{I} shall be an integral type or meet the requirements of a forward iterator type; otherwise, \tcode{I} shall be an integral type or meet the requirements of an input iterator type. \tcode{Size} shall be an integral type and \tcode{n} shall be non-negative. \tcode{S} shall have integral type and \tcode{stride} shall have non-zero value. \tcode{stride} shall be negative only if \tcode{I} has integral type or meets the requirements of a bidirectional iterator. The \tcode{rest} parameter pack shall have at least one element, comprising objects returned by invocations of \tcode{reduction} ([parallel.alg.reduction]) and/or \tcode{induction} ([parallel.alg.induction]) function templates followed by exactly one invocable element-access function, {\em f}. For the overloads with an \tcode{ExecutionPolicy}, {\em f} shall meet the requirements of \tcode{CopyConstructible}; otherwise, {\em f} shall meet the requirements of \tcode{MoveConstructible}.

\pnum \effects Applies {\em f} to each element in the \defn{input sequence}, as described below, with additional arguments corresponding to the reductions and inductions in the \tcode{rest} parameter pack. The length of the input sequence is:

\begin{itemize}
\item \tcode{n}, if specified,

\item otherwise \tcode{finish - start} if neither \tcode{n} nor \tcode{stride} is specified,

\item otherwise \tcode{1 + (finish-start-1)/stride} if \tcode{stride} is positive,

\item otherwise \tcode{1 + (start-finish-1)/-stride}.
\end{itemize}

The first element in the input sequence is \tcode{start}. Each subsequent element is generated by adding \tcode{stride} to the previous element, if \tcode{stride} is specified, otherwise by incrementing the previous element.
\begin{note}As described in the \Cpp standard, section [algorithms.general], arithmetic on non-random-access iterators is performed using advance and distance.\end{note}
\begin{note}The order of the elements of the input sequence is important for determining ordinal position of an application of {\em f}, even though the applications themselves may be unordered.\end{note}

The first argument to {\em f} is an element from the input sequence.\begin{note}If \tcode{I} is an iterator type, the iterators in the input sequence are not dereferenced before being passed to {\em f}.\end{note} For each member of the \tcode{rest} parameter pack excluding {\em f}, an additional argument is passed to each application of {\em f} as follows:

\begin{itemize}
\item If the pack member is an object returned by a call to a reduction function listed in section [parallel.alg.reductions], then the additional argument is a reference to an accumulator of that reduction object.

\item If the pack member is an object returned by a call to \tcode{induction}, then the additional argument is the induction value for that induction object corresponding to the position of the application of {\em f} in the input sequence.
\end{itemize}

\pnum \complexity Applies {\em f} exactly once for each element of the input sequence.

\pnum \remarks If {\em f} returns a result, the result is ignored.

\end{itemdescr}

\rSec2[parallel.alg.novec]{No vec}

\begin{itemdecl}
template<class F>
  auto no_vec(F&& f) noexcept -> decltype(std::forward<F>(f)());
\end{itemdecl}

\begin{itemdescr}
\pnum \effects Evaluates \tcode{std::forward<F>(f)()}. When invoked within an element access function in a parallel algorithm using \tcode{vector_policy}, if two calls to \tcode{no_vec} are horizontally matched within a wavefront application of an element access function over input sequence S, then the execution of \tcode{f} in the application for one element in S is sequenced before the execution of \tcode{f} in the application for a subsequent element in S; otherwise, there is no effect on sequencing.

\pnum \returns The result of \tcode{f}.

\pnum \realnotes If \tcode{f} exits via an exception, then \tcode{terminate} will be called, consistent with all other potentially-throwing operations invoked with \tcode{vector_policy} execution.

\begin{example}
\begin{codeblock}
extern int* p;
for_loop(vec, 0, n[&](int i) {
  y[i] +=y[i+1];
  if(y[i] < 0) {
    no_vec([]{
      *p++ = i;
    });
  }
});
\end{codeblock}

The updates \tcode{*p++ = i} will occur in the same order as if the policy were \tcode{seq}.
\end{example}

\end{itemdescr}

\rSec2[parallel.alg.ordupdate.class]{Ordered update class}

\begin{codeblock}
template<class T>
class ordered_update_t {
  T& ref_; // exposition only
public:
  ordered_update_t(T& loc) noexcept
    : ref_(loc) {}
  ordered_update_t(const ordered_update_t&) = delete;
  ordered_update_t& operator=(const ordered_update_t&) = delete;

  template <class U>
    auto operator=(U rhs) const noexcept
      { return no_vec([&]{ return ref_ = std::move(rhs); }); }
  template <class U>
    auto operator+=(U rhs) const noexcept
      { return no_vec([&]{ return ref_ += std::move(rhs); }); }
  template <class U>
    auto operator-=(U rhs) const noexcept
      { return no_vec([&]{ return ref_ -= std::move(rhs); }); }
  template <class U>
    auto operator*=(U rhs) const noexcept
      { return no_vec([&]{ return ref_ *= std::move(rhs); }); }
  template <class U>
    auto operator/=(U rhs) const noexcept
      { return no_vec([&]{ return ref_ /= std::move(rhs); }); }
  template <class U>
    auto operator%=(U rhs) const noexcept
      { return no_vec([&]{ return ref_ %= std::move(rhs); }); }
  template <class U>
    auto operator>>=(U rhs) const noexcept
      { return no_vec([&]{ return ref_ >>= std::move(rhs); }); }
  template <class U>
    auto operator<<=(U rhs) const noexcept
      { return no_vec([&]{ return ref_ <<= std::move(rhs); }); }
  template <class U>
    auto operator&=(U rhs) const noexcept
      { return no_vec([&]{ return ref_ &= std::move(rhs); }); }
  template <class U>
    auto operator^=(U rhs) const noexcept
      { return no_vec([&]{ return ref_ ^= std::move(rhs); }); }
  template <class U>
    auto operator|=(U rhs) const noexcept
      { return no_vec([&]{ return ref_ |= std::move(rhs); }); }

  auto operator++() const noexcept
    { return no_vec([&]{ return ++ref_; }); }
  auto operator++(int) const noexcept
    { return no_vec([&]{ return ref_++; }); }
  auto operator--() const noexcept
    { return no_vec([&]{ return --ref_; }); }
  auto operator--(int) const noexcept
    { return no_vec([&]{ return ref_--; }); }
};
\end{codeblock}

\pnum An object of type \tcode{ordered_update_t<T>} is a proxy for an object of
type \tcode{T} intended to be used within a parallel application of an element
access function using a policy object of type \tcode{vector_policy}. Simple
increments, assignments, and compound assignments to the object are forwarded
to the proxied object, but are sequenced as though executed within a \tcode{no_vec}
invocation. \begin{note}The return-value deduction of the forwarded operations
results in these operations returning by value, not reference. This formulation
prevents accidental collisions on accesses to the return value.\end{note}

